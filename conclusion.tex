\section{Conclusion} \label{sec:conclusion}
In this paper, grounded in data collected from successfully DevOps adoption
experiences, we present a theory on DevOps adoption, a model of how to adopt
DevOps according with this theory and a case where this model has been tested
in practice.

We found that the DevOps adoption involves a very specific relationship between
eight key categories: agility, automation, collaboration culture, continuous
measurement, quality assurance, resilience, sharing, and transparency. The core
category of DevOps adoption is the collaboration culture. Some of the
identified categories (i.e., automation, sharing, and transparency) only exist
to propitiate the foundation of a collaboration culture. Other categories
(i.e., agility and resilience) are expected consequences of this formation.
Finally, two other categories (i.e., continuous measurement and quality
assurance) work as both foundations and consequences. We call the foundations
categories as ``DevOps enablers'', and the consequences categories as ``DevOps outcomes''.

Crucially, we propose that this way of guidance simplify the understanding of the
complex set of elements that are part of DevOps adoption, enabling it to be
more direct and with lower risk of focusing on wrong things.
