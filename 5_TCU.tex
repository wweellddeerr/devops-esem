\section{Application of The Model} \label{sec:TCU}

{
\color{blue}
In this section we detail an real experience on DevOps adoption where we could
demonstrate the practical application of the model proposed above. Initially,
we explain the context of the institution in which the model was applied; after
we present, step by step, our perception about the model application; and,
finally, we present the results of a focus group containing the perception of
four professionals from the company about the adoption of DevOps as a whole,
including their perception about the applicability and utility of our model
during the process.
}

\subsection{Context}

The TCU is responsible for the accounting, financial, budget, performance, and property
oversight of federal institutions and entities of the country. Currently, there are 2500
professionals working at the TCU, of which approximately 300 work directly on either
software development or operations. The source code repository at the TCU hosts
more than 200 software project, totaling over 4 million lines of code.

{
\color{blue}
For many years, the TCU software development process was marked by a strong call
for standardization of procedures. This approach favored the specialization of
activities, causing segregation of development and operations teams in a rigid
silo structure. Work on this structure revolved around bureaucratic
communication and well-defined service level agreements - SLAs. Communication
between teams occurred basically through the use of a corporate service-desk tool.
The SLAs provided maximum deadlines for the operations team to perform the
procedures requested by the development team through the service-desk. The
procedures include provisioning of infrastructure and database servers for
development, testing, staging and production environments, requesting access
to server logs, and the most controversial of all: deployment of new versions
of applications. The deployment involved a long chain of a week-long process,
whose complex chaining of responsibilities often ran out of schedule, causing
months of delays in software deliveries.

In addition to delays in software deliveries, the silo structure caused a
devastating blaming game between the teams. If there were delays in delivery
or errors in some functionality, the focus was not on solving the problem, but
on pointing out which team was the culprit.

In search of solutions to these problems, and knowing the alleged benefits of
the approach, the TCU included the development of the DevOps approach among the
objective of its IT area. The first attempt by the technical teams to reach the
goal was to seek consulting firms whose proposed solutions invariably involved
implanting specific DevOps tools. These DevOps tools were typically related to
automate some aspect of the process and, naturally, the teams turned their
attentions to establish clear and separated responsibilities to each aspect of
the tools. Obviously, these DevOps tools only changed the points of conflict.
In this context, our research was developed and, after that, we applied our
model aiming to increase the DevOps level at TCU. In this section we described
this experience.
}

\subsection{Applying The Model}

Here we present our perception about how our model has contributed to DevOps
adoption at TCU.

\subsubsection{Disseminating the Collaborative Culture}

Before the application of our model, the TCU had produced some w.r.t deployment
automation results and the focus was being directed to the tooling issue.
Considering this incomplete perspective of DevOps, the conflicts between
development and operations teams continued. That is, the mere advance in
implanting ``DevOps tools'' simply changed the points of conflict, but they
persisted.

{
\color{blue}
To start disseminating our model at TCU, we conducted a series of lectures to
explain both the model itself and the way it was formulated. After knowing the
model, the professionals began to understand that if the TCU wants to succeed in
DevOps adoption, tooling and automation would not be enough. We consider that
the teams changed their focus to build a \cc. This change was only possible due
two factors: (1) the engagement and sponsorship of the IT managers and (2) the
explanations about the process of constructing the model: although the TCU is a
governmental institution, its IT professionals recognize the importance of
staying in line with industry practice to avoid an already before faced weight
of dealing with completely legacy software.

Looking to the concepts within the \cc category, the first practical action at
the TCU was to facilitate communication between teams. The use of tickets and
SLAs were then abolished in most of the scenarios. In its turn, the concepts of
\textbf{software development empowerment} and \textbf{shared responsibilities}
have found some resistance from some professionals. The strategy to bypass this
kind of cultural resistance involves the use of enablers that will be discussed
bellow and awareness about: (1) fortifying the collaboration will bring benefits
to all involved and especially to the TCU and (2) following a model builded upon
well succeeded experiences is a low-risk strategy for TCU to explore the
benefits of DevOps.

Finally, we consider that the efforts to disseminate and promote the \cc are
continuous, but the kick-off was given and it is possible to notice that the
idea has been widely accepted and applied by most of the involved professionals.

}

\subsubsection{Applying the Enablers}

Looking to enablers, the TCU is applying \cat{sharing and transparency} concepts.
The role of internal tech talks and committees to disseminate that collaboration
culture and related concepts is being reinforced.
When a new infrastructure had to be provided and configured, the current guideline is
that there must be a kind of \emph{pair programming} between developers and infrastructure
members. All application related tasks must be executed in a collaborative
way. Naturally, the professionals noticed that automation would facilitate the
operationalization of that collaboration. For this reason, the infrastructure provisioning
and management was automated.

The TCU also uses continuous measurement and quality assurance concepts as
enablers of its DevOps adoption. The applications started to be continuously
tested and measured. The tests were automated and included in the pipelines.
Verification of test coverage and quality code also became part of the pipeline.
This increased the confidence between teams. The TCU started
to explore the potential of DevOps tools, like recovery automation, zero
down-time, and auto scaling. The deployment has also been automated.
It is important to note that, before DevOps, deployment activities were historically a controversial point at the TCU.
Several conflicts occurred over time. Rigid procedures were created to try to
avoid problems. These ``rigid procedures'' often led to periods of months
without any software delivery. The more collaborative scenario, with a strong appeal in automation and quality,
created by following an appropriate path in adopting DevOps, enabled the deployment activities to become
a lightweight task at the TCU. Continuous deployment became a reality and, currently, several deployments
occur as regular activities of the development teams at the TCU.

Since the TCU is a government institution, some advances in DevOps adoption still comes up
against regulatory issues. For example, there are internal regulations that
establish that only the operations sector is responsible for issues related to
application infrastructure, contrasting with shared responsibilities that are
part of the \cc. Nevertheless, our model enabled the TCU to adopt DevOps in a more
sustainable way. Knowing the role
of each DevOps element in the adoption was fundamental for the TCU to avoid points
of failure and to build a collaborative environment that supports the
exploration of DevOps benefits.

\subsubsection{Checking the Outcomes}

\subsection{A Focus Group on DevOps Adoption at TCU}

{
\color{blue}
To reduce the bias of our perception about the scenario of adopting DevOps at
TCU, we conducted a focus group with four professionals of the company for an
empirical evaluation about their perception about the adoption of DevOps.

Focus group emerged as a research method in the social sciences in the 1950s
and is currently widely used, for example, in sociological studies, market
research, product planning, and system usability studies~\cite{shull2007guide}.
Morgan~\cite{morgan1996focus} defines focus group as a research technique that
collects data through group interaction on a specific topic determined by the
researcher.

According to F. Shull et al.~\cite{shull2007guide}, focus groups typically have
between three and twelve participants, are designed to obtain personal
perceptions of members of one or more groups involved in a defined area of
research interest and have as benefits the production of candid, often
insightful information, with a low cost and fast execution. These
characteristics make the focus group an adequate alternative to the purposes
of this evaluation. According to the authors, the discussion is guided and
facilitated by a researcher-moderator who follows a predefined structure of
questions.

The focus group at TCU was attended by four professionals, two of each
development and operations team. The profile of participants is described in
Table \ref{table_participants_focus_group}.

\begin{table}[hb!]
\centering
\label{table_participants_focus_group}
\begin{tabular}{|p{0.4cm}|p{0.8cm}|p{1.8cm}|p{4cm}|} \hline
{\bf P\#} & {\bf Team} & {\bf Formation} & {\bf Experience}\\ \hline
P1 & Dev & Graduate & 3 years in dev team at TCU and 9 years of previous experience \\ \hline
P2 & Dev & Posgraduate & 6 years in dev team at TCU and 7 years of previous experience \\ \hline
P3 & Ops & Graduate & 3 years in dev team at TCU and 8 years of previous experience \\ \hline
P4 & Ops & Graduate & 3 years in dev team at TCU and 10 years of previous experience \\ \hline
\end{tabular}
\caption{Focus Group Participants}
\end{table}

Following a similar structure to that performed by Lehtola et al.~\cite{requirementes_priorization_in_practice},
the focus group was conducted as follows: (1) the researcher-moderator served as
focus group facilitator providing participants with three discussion topics
listed in Table \ref{table_topics}; (2) at the beginning of the discussion of each topic, the
questions were presented to participants who wrote their ideas and keywords in
post-it note and (3) after that, the notes were placed on a white board and served
as a starting point for discussions on the respective topic in order to reach
conclusions about the respective question. The results of the discussions of
each topic are presented below.

\begin{table}[hb!]
\centering
\label{table_topics}
\begin{tabular}{|p{0.2cm}|p{3.4cm}|p{3.8cm}|} \hline
& \textbf{Topic} & \textbf{Questions} \\ \hline

1 & Current status of DevOps adoption at TCU &
1. What actions developed in the TCU do you consider to be part of DevOps adoption?\newline\newline
2. What previously existing problems have been solved by these actions? \\ \hline

2 & Applicability and utility of the proposed model &
1. Do you consider that the proposed model has contributed to DevOps adoption at TCU?\newline\newline
2. If so, what are the main contributions? \\ \hline

3 & Challenges faced and next steps in DevOps adoption &
1. What are the main challenges that TCU currently faces in DevOps adoption?\newline\newline
2. What are the next steps in DevOps adoption at TCU?\\ \hline

\end{tabular}
\caption{Focus Group Topics}
\end{table}
}

\subsubsection{Current Status of DevOps Adoption at TCU}

{
\color{blue}
The first action indicated and discussed in the group was the \textbf{provision of
environments} (Virtual Machines) for the installation of tools that are related
to the development work. This action was exemplified by the successful
installations of the Elasticsearch\footnote{https://www.elastic.co/} and
Kafka\footnote{https://kafka.apache.org/} tools. The previously existing problem
was that when a developer needed a tool like these, inherent in his work and
necessary to adequately address specific problems, he depended on opening a
request for the operations team to provide it, with deadlines that often make
the use of the most adequate solution unfeasible. With VM provisioning and
cooperation between the two teams, these tools became available quickly for use
and the responsibility for its management is joint. This is a clear example of
application of the concepts of \textbf{software development empowerment} and
\textbf{shared responsibility} of the core category \cc.

Next, the use of \textbf{microservices}, and \textbf{containers} as environment
to run them was debated. The first problem that this action solved, in the
participants' understanding, was the previous lack of parity between the environments
(development, test, staging and production). The recurring problems of
applications that worked in a development environment but which had problems
in production were recalled, which was solved with the use of containers. In
this context, the use of tools like Docker\footnote{https://www.docker.com/}
and Kubernetes\footnote{https://kubernetes.io} was discussed as means of providing
configuration details of the environments in the applications source code
repositories themselves. This enabled both development and operations teams to
get a first idea of the running environment of each application in a more
transparent way. Finally, the use of containers and related tools also enabled
the use of mechanisms for horizontal scalability, high availability and
publication of applications without down-time, solving a recurring problem of
interrupting the work of business teams during the deployment of the applications.
Here, it is possible to identify several concepts of our theory, such as:
(1) parity between environments, (2) infrastructure provisioning automation,
(3) autonomous services, (4) containerization, (5) auto scaling, (6) recovery
automation and (7) zero down-time.

The third discussed point was the \textbf{reduction of bureaucracy in communication}
between the teams. It was pointed out that although there is still much room
for progress, this can already be considered as one of the advances of the TCU
related to DevOps adoption. During the discussion, the extremely ceremonious
communication process in the scope of the deployment SLA was recalled. There is
a current guideline to avoid using service-desk for simple problem solving. The
problems had to be solved in a collaborative way, preferably face to face and
the use of the Slack tool has been institutionalized and facilitated the contact
between the two teams. We highlight the concept \textbf{straightforward communication}
of the \cc category as part of this point of discussion.

Then, the use of \textbf{multidisciplinary pipelines} in the most recent applications of
TCU was pointed out and debated. These pipelines involve everything from the
build, through automated tests and static analysis of source code, execution of
containers using Kubernetes and publication isonomically in the different
environments (development, staging and production). A Jenkins is used as tool
to describe and execute the pipelines. One single trigger execute a set of
steps that previously required several comings and goings between the teams and
a long time to complete. The group agreed that the construction of multidisciplinary
and collaboratively produced and maintained pipelines like these ones is only
possible when the \cc is fostered. This point of discussion refers to a few
more concepts: (1) operations in day-to-day development, (2) test automation,
(3) deployment automation, (4) shared pipelines, (5) continuous integration,
(6) continuous infrastructure provisioning, (7) continuous deployment,
(8) continuous testing and (9) source code static analysis.

\textbf{Automated Database Migrations} was the next point considered as part of
DevOps adoption. The participants explored the differences between the previous
scenario - where changes in the database structure of an application needed to be
made according to an SLA and, therefore, requested through service-desk - and
the new one where the Flyway\footnote{https://flywaydb.org/} tool is used to
manage database migrations. The use of these type of tool had previously been
discarded because the operations team could not provide the database owner
password. This discussion was retaken recently, and the teams jointly developed
a solution to safely share the owner password. This discussion reinforces two
aspects of the \cc that are the confidence and the collaboration between the
teams. The concept of \textbf{infrastructure management automation} was present
here.

Finally, the last point considered by the group as part of DevOps adoption was
one more solution built in a collaborative way for \textbf{continuously and automated
monitoring} of application errors. Previously, a developer's simple access
to an application's log needed to be requested through the service-desk. The
solution automatically collects the logs, searches for errors in its content,
and send messages through Slack to both teams in case of error. We can find
the application of two other concepts: (1) monitoring automation and
(2) application log monitoring.

}

\subsubsection{Applicability and Utility of the Proposed Model}

\subsubsection{Challenges Faced and Next Steps in DevOps Adoption}

\subsubsection{General Considerations}

Even though it was not the only point of debate, considerations about the model
permeated all the discussed topics, practical actions were highlighted that only
materialized due to the exchange of experiences that occurred during this
research. Many of the concepts presented in the model were visualized during
focus group discussions, this is not a mere coincidence and emphasizes that our
model is guiding, in a high level of abstraction, the actions of the TCU toward
DevOps adoption. In addition, it was possible to note a previously non-existent
concern about the \cc, the participants frequently placed their actions as part
of the efforts to foster the \cc.
