\section{Application of The Model} \label{sec:TCU}

{
\color{blue}
In this section we detail a real experience on DevOps adoption where we could
demonstrate the practical application of our DevOps adoption model. Initially,
we explain the context of the institution where we applied our model; after that, 
we present our perceptions about the model application; and,
finally, we present the results of a focus group containing the perception of
four professionals from the institution about the adoption of DevOps as a whole,
including their opinion about the applicability and utility of our model
during the process.
}

\subsection{Context}

TCU is responsible for the accounting, financial, budget, performance, and property
oversight of federal institutions and entities of the country. Currently, there are 2500
professionals working at TCU, of which approximately 300 work directly on either
software development or operations. The source code repository at TCU hosts
more than 200 software projects, totaling over 4 million lines of code.

{
\color{blue}
For many years, the software development process at TCU was characterized by a strong call
for standardization of procedures. This approach favored the specialization of
activities, causing segregation of development and operations teams in a rigid
silo structure. Work on this structure revolved around bureaucratic
communication and well-defined service level agreements (SLAs). Communication
between teams occurred basically through the use of a corporate service-desk tool.
The SLAs mostly ensured deadlines for the operations team to perform the
procedures that the development teams requested through the service-desk. The
procedures include provisioning of infrastructure and database servers for
development, testing, staging, and production environments, requesting access
to server logs, and the most controversial of all: deployment of new versions
of applications. The deployment involved a long chain of a week-long process,
whose complex chaining of responsibilities often ran out of schedule, causing
months of delays in software deliveries.

In addition to delaying software delivering, the silo structure caused a
devastating blaming game between the teams. If there were delays in delivery
or errors in some functionality, the focus was not on solving the problem, but
rather on pointing out which team was responsible for the problem.

In search of solutions to these problems, and knowing the alleged benefits of
the approach, TCU included the development of the DevOps approach among the
objective of its IT area. The first attempt by the technical teams to reach the
goal was to hire consulting firms whose proposed solutions invariably involved
implanting specific DevOps tools. These DevOps tools were typically related to
automate some aspect of the process and, naturally, the teams turned their
attentions to establish clear and separated responsibilities to each aspect of
the tools. Obviously, these DevOps tools only changed the points of conflict.
In this context, our research was developed and, after that, we applied our
model aiming to increase the DevOps level at TCU. In this section we described
this experience.
}

\subsection{Applying The Model}

Here we present our perception about how our model has contributed to DevOps
adoption at TCU.

\subsubsection{Disseminating the Collaborative Culture}

Before applying our model, TCU had been using automation for supporting deployment
activities and focusing on the tooling dimention of DevOps.
Considering this incomplete perspective, the conflicts between
development and operations teams continued. That is, the mere advance in
implanting ``DevOps tools'' simply changed the points of conflict, but they
persisted.

{
\color{blue}
To start disseminating our model at TCU, we conducted a series of lectures to
explain both the model itself and the way it was formulated. After knowing the
model, the professionals began to understand that if TCU wants to succeed in
DevOps adoption, tooling and automation would not be enough. We consider that
the teams changed their focus to build a \cc. This change was only possible due
to two aspects: (1) the engagement and sponsorship of the IT managers and (2) the
explanations about the process of constructing the model: although TCU is a
governmental institution, its IT professionals recognize the importance of
staying in line with industry practice to avoid an already before faced weight
of dealing with completely legacy software.

Looking to the concepts within the \cc category, the first practical action at
TCU was to facilitate communication between teams. The use of tickets and
SLAs were then abolished in most of the scenarios. In its turn, the concepts of
\textbf{software development empowerment} and \textbf{shared responsibilities}
have found some resistance from some professionals. The strategy to mitigate this
cultural resistance involved the use of enablers (that will be further discussed)
and awareness about: (1) fortifying the collaboration will bring benefits
to all involved and especially to TCU and (2) following a model built upon
well succeeded experiences is a low-risk strategy for TCU to explore the
benefits of DevOps.

Finally, we consider that the efforts to disseminate and promote the \cc are
continuous, but the kick-off was given and it is possible to notice that the
idea has been widely accepted and applied by most of the involved professionals.

}

\subsubsection{Applying the Enablers}

Considering the enablers dimention of our model, TCU is currently
applying \cat{sharing and transparency} concepts.
The role of internal tech talks and committees to disseminate that \cc
and related concepts is being reinforced. When a new infrastructure had
to be provided and configured, the current guideline is that these
activities should be conducted in ``pairs'' involving members of
development and infrastructure teams. All application related tasks must be executed in a collaborative
way. Naturally, the professionals noticed that automation would facilitate the
operationalization of that collaboration. For this reason, the infrastructure
provisioning and management was automated.

TCU also uses continuous measurement and quality assurance concepts as
enablers of its DevOps adoption. The applications started to be continuously
tested and measured. The tests were automated and included in the pipelines.
Verification of test coverage and quality code also became part of the pipeline.
This increased the confidence between teams. {\color{blue} More confidence leads to a more
robust \cc, and a more robust \cc enables the company to explore the benefits
of DevOps as will be seen below when checking the outputs.

It is important to note that, before DevOps, deployment activities were
historically a controversial point at the TCU. Several conflicts occurred over
time. Rigid procedures were created to try to avoid problems. These ``rigid
procedures'' often led to periods of months without any software delivery. As
our model advocates that there is no precedence between enablers, the specific
characteristics of the company are a good starting point to select the most
relevant enablers to explore. So, the TCU' teams started by the deployment,
applying automation and quality checking in the process. With the automated
execution of tests and static source code analysis, the development team
demonstrates to the operations team that the products had quality, increasing
confidence between them and reducing the necessity of rigid procedures to put
some software in production. So, to automate the deployment in a collaborative
way, the teams jointly developed an infrastructure as code strategy to put the
details about deployment in the application source code repository itself,
increasing the transparency. Little by little the deployment of applications has
ceased to be a ceremonial event to become a day-to-day activity.
}

\subsubsection{Checking the Outcomes}

Initially, we highlight that a more robust \cc enabled TCU to explore a scenario
of continuous deployment of some applications. The newer applications have
pipelines that publish every commit in production. This continuous delivery is
only possible in applications whose pipelines contain several quality checks,
mitigating the risk of something going wrong. Additionally, automated monitoring
is required for these applications. Some applications have a history of dozen
of deployments in a single day, contrasting with one deployment per week in the
best case, or more than a month without publishing in the worst.

To reduce even more the risks related to continuously delivery software in
production, and taking advantage of greater collaboration between its Development
and operations teams, the TCU started to explore the potential of DevOps tools,
like recovery automation, zero down-time, and auto scaling.

In our evaluation, the TCU' change of focus from tooling to collaboration was
decisive to increase the confidence between teams, reduce the blame game and,
consequently, to enable the company to sustainably exploit the benefits of
adopting DevOps. In addition, our model has served as a roadmap that allows
the teams to focus on the \cc. So, the outcomes already present in TCU' DevOps
Adoption are compatible with those foreseen in the third step of our model.

\subsubsection{General Considerations}
Since the TCU is a government institution, some advances in DevOps adoption
still comes up against regulatory issues. For example, there are internal
regulations that establish that only the operations sector is responsible for
issues related to application infrastructure, contrasting with shared
responsibilities that are part of the \cc.

{
\color{blue}
The \cc need to be continuously fomented and our model is only a general roadmap,
the efforts need to continue over time to deal with incidents that disfavor
the collaboration and to restrain the emergence of new ones.
}

Nevertheless, our model enabled the TCU to adopt DevOps in a more sustainable
way. Knowing the role of each DevOps element in the adoption was fundamental for
the TCU to avoid points of failure and to build a collaborative environment that
supports the exploration of DevOps benefits.

\subsection{A Focus Group on DevOps Adoption at TCU}

{
\color{blue}
To reduce the bias of our perception about the scenario of adopting DevOps at
TCU, we conducted a Focus Group with four professionals of the company for an
empirical evaluation of their perception about DevOps adoption.

The focus group at TCU lasted approximately 3 hours and was attended by four
professionals, two of each development and operations team. The participants'
profile is described in Table \ref{focusgroup_part}.

\begin{table}[hb!]
\centering
\begin{tabular}{|p{0.6cm}|p{1.2cm}|p{2.5cm}|p{6cm}|} \hline
{\bf P\#} & {\bf Team} & {\bf Formation} & {\bf Experience}\\ \hline
P1 & Dev & Graduate & 3 years in dev team at TCU and 9 years of previous experience \\ \hline
P2 & Dev & Posgraduate & 6 years in dev team at TCU and 7 years of previous experience \\ \hline
P3 & Ops & Graduate & 3 years in ops team at TCU and 8 years of previous experience \\ \hline
P4 & Ops & Graduate & 3 years in ops team at TCU and 10 years of previous experience \\ \hline
\end{tabular}
\caption{Focus Group Participants}
\label{focusgroup_part}
\end{table}

We approached three discussion topics during the focus group. These topics are
listed in Table \ref{table_topics} and its results are presented below.

\begin{table}[hb!]
\centering
\begin{tabular}{|p{0.2cm}|p{3.4cm}|p{10cm}|} \hline
& \textbf{Topic} & \textbf{Questions} \\ \hline

1 & Current status of DevOps adoption at TCU &
1. What actions developed in the TCU do you consider to be part of DevOps adoption?\newline\newline
2. What previously existing problems have been solved by these actions? \\ \hline

2 & Applicability and utility of the proposed model &
1. Do you consider that the proposed model has contributed to DevOps adoption at TCU?\newline\newline
2. If so, what are the main contributions? \\ \hline

3 & Challenges faced and next steps in DevOps adoption &
1. What are the main challenges that TCU currently faces in DevOps adoption?\newline\newline
2. What are the next steps in DevOps adoption at TCU?\\ \hline

\end{tabular}
\caption{Focus Group Topics}
\label{table_topics}
\end{table}
}

\subsubsection{Current Status of DevOps Adoption at TCU}

{
\color{blue}
The first action indicated and discussed in the group was the \textbf{provision of
environments} (Virtual Machines) for the installation of tools that are related
to the development work. This action was exemplified by the successful
installations of the Elasticsearch\footnote{https://www.elastic.co/} and
Kafka\footnote{https://kafka.apache.org/} tools. The previously existing problem
was that when a developer needed a tool like these, inherent in his work and
necessary to adequately address specific problems, he depended on opening a
request for the operations team to provide it, with deadlines that often make
the use of the most adequate solution unfeasible. With VM provisioning and
cooperation between the two teams, these tools became available quickly for use
and the responsibility for its management is joint. This is a clear example of
application of the concepts of \textbf{software development empowerment} and
\textbf{shared responsibility} of the core category \cc.

Next, the use of \textbf{microservices}, and \textbf{containers} as environment
to run them was debated. The first problem that this action solved, in the
participants' understanding, was the previous lack of parity between the environments
(development, test, staging and production). The recurring problems of
applications that worked in a development environment but which had problems
in production were recalled, which was solved with the use of containers. In
this context, the use of tools like Docker\footnote{https://www.docker.com/}
and Kubernetes\footnote{https://kubernetes.io} was discussed as means of providing
configuration details of the environments in the applications source code
repositories themselves. This enabled both development and operations teams to
get a first idea of the running environment of each application in a more
transparent way. Finally, the use of containers and related tools also enabled
the use of mechanisms for horizontal scalability, high availability and
publication of applications without down-time, solving a recurring problem of
interrupting the work of business teams during the deployment of the applications.
Here, it is possible to identify several concepts of our theory, such as:
(1) parity between environments, (2) infrastructure provisioning automation,
(3) autonomous services, (4) containerization, (5) auto scaling, (6) recovery
automation and (7) zero down-time.

The third discussed point was the \textbf{reduction of bureaucracy in communication}
between the teams. It was pointed out that although there is still much room
for progress, this can already be considered as one of the advances of the TCU
related to DevOps adoption. During the discussion, the extremely ceremonious
communication process in the scope of the deployment SLA was recalled. There is
a current guideline to avoid using service-desk for simple problem solving. The
problems had to be solved in a collaborative way, preferably face to face and
the use of the Slack tool has been institutionalized and facilitated the contact
between the two teams. We highlight the concept \textbf{straightforward communication}
of the \cc category as part of this point of discussion.

Then, the use of \textbf{multidisciplinary pipelines} in the most recent applications of
TCU was pointed out and debated. These pipelines involve everything from the
build, through automated tests and static analysis of source code, execution of
containers using Kubernetes and publication isonomically in the different
environments (development, staging and production). A Jenkins is used as tool
to describe and execute the pipelines. One single trigger execute a set of
steps that previously required several comings and goings between the teams and
a long time to complete. The group agreed that the construction of multidisciplinary
and collaboratively produced and maintained pipelines like these ones is only
possible when the \cc is fostered. This point of discussion refers to a few
more concepts: (1) operations in day-to-day development, (2) test automation,
(3) deployment automation, (4) shared pipelines, (5) continuous integration,
(6) continuous infrastructure provisioning, (7) continuous deployment,
(8) continuous testing and (9) source code static analysis.

\textbf{Automated Database Migrations} was the next point considered as part of
DevOps adoption. The participants explored the differences between the previous
scenario - where changes in the database structure of an application needed to be
made according to an SLA and, therefore, requested through service-desk - and
the new one where the Flyway\footnote{https://flywaydb.org/} tool is used to
manage database migrations. The use of these type of tool had previously been
discarded because the operations team could not provide the database owner
password. This discussion was retaken recently, and the teams jointly developed
a solution to safely share the owner password. This discussion reinforces two
aspects of the \cc that are the confidence and the collaboration between the
teams. The concept of \textbf{infrastructure management automation} was present
here.

Finally, the last point considered by the group as part of DevOps adoption was
one more solution built in a collaborative way for \textbf{continuously and automated
monitoring} of application errors. Previously, a developer's simple access
to an application's log needed to be requested through the service-desk. The
solution automatically collects the logs, searches for errors in its content,
and send messages through Slack to both teams in case of error. We can find
the application of two other concepts: (1) monitoring automation and
(2) application log monitoring.

}

\subsubsection{Applicability and Utility of the Proposed Model}

{
\color{blue}
All participants of the focus group agreed that the proposed model has great
utility in DevOps adoption at TCU. They remembered that most of the actions
discussed in the previous topic were direct result of the model development and,
therefore, its application is already being effective and producing results in
expanding DevOps usage throughout the development of TCU' enterprise applications.
The following are the two main benefits of model usage, discussed in the focus
group:

\textbf{DevOps Institutional Understanding}: In response to the question about
the model' contributions to TCU, initially it was pointed out that, during the
initial attempts with consultant firms, it was clear that the mere use of
tools did not bring the teams closer together. Some developers acted as if DevOps
had given them permission to ignore operations team' procedures. And the
operations team was overly concerned in formally delimit the administration
responsibilities to each used tool. The discussion advance showed that they all
agreed that fostering a \cc was not taken into account before. And seeing,
throughout the model, that the DevOps adoption in well succeeded scenarios goes
mainly through this point, has made possible a change in teams posture about
collaboration.

Subsequently, a post-it note about the ``wide range of practices and experiences''
present in the model was discussed. It was once again recalled that several
practices have already been implemented using as input industry experiences
collected during model production. The model has also been pointed out as a tool
for evaluating practices that the TCU does not yet adopt, providing a road map
to guide next steps.

\textbf{Industry Experiences}: Finally, it was emphasized that the model was
built taking into account successful industry experiences and this represents
great value for the TCU. In the group's understanding, although the TCU
possesses many governmental environments peculiarities, the search for
technological innovation is part of its strategic map, and can not be achieved
by looking only at scenarios similar to the current one. It was emphasized that
the industry is an important player in the definition of new technologies that,
adapted to a greater or lesser degree, may be fully applicable to government
agencies, such as TCU. According to the formed understanding, the fact that this
model, built upon industry experiences, is already being effectively applied, is
one more confirmation that government agencies can effectively innovate following
industry tendencies.
}

\subsubsection{Challenges Faced and Next Steps in DevOps Adoption}
{
\color{blue}

The discussions of the last focus group topic focused on identifying the
challenges faced during the evolution of DevOps usage at TCU, as well as the
next steps to overcome the challenges and institutionalize DevOps as a
software development approach.

\textbf{DevOps Internal Understanding Maturity}:
It was initially debated the perception pointed out by one of the participants
that there is still a lot of hype around what would be DevOps adoption. According
to him, some developers still thinking that DevOps allows them to take technical
initiatives without consulting other professionals, and that some operations
people still do not feel comfortable with this paradigm shift because they
understand that DevOps can cause disorganization in an environment that already
had stability. It was mentioned that the model helps to deal with this challenge,
but that a permanent effort is needed to foment the collaboration between the teams,
avoiding someone to leave wanting to solve everything according to personal
convictions.

It was also pointed out as a challenge, the difficulty of disseminating
knowledge related to new tools and processes that came along with DevOps adoption.
Actions to mitigate this challenge have been discussed, including the expansion
of internal lectures, participation in events such as DevOpsDays, and the stimuli
that the TCU already offers to its professionals, such as training license, refund
of training, and availability of one online training platform. In this sense,
it was understood that one of the next steps is the expansion of the technical
capacity of the professionals in themes related to the modernization of tools
and processes.

\textbf{Information Security}:
Here, the group discussed that DevOps adoption has considerably increased the
TCU's surface of technological vulnerabilities. It was pointed out that the
operations team is very concerned about information security, and that its
professionals are evaluating the implemented tools and will propose modifications.
The participants then aligned that this debate can not be only on the operations
team, as this is a manifestation of collaboration lack.

The P2 then suggested that these concerns extend the scope from DevOps to a
DevSecOps context, when security activities are also integrated into the
development process. There is then a further step, which is to extend the
DevSecOps perspective.

\textbf{Metrics Collection in Applications}:
In this part of the discussion, the participants discussed that the current
continuous monitoring solution is restricted to application errors, and that
the model contains ideas about collecting metrics in applications to foster
business decisions and applications evolutions. The P4 pointed out that the same
solution can be extended as long as the applications were instrumented to
generate logs of any other metrics. Therefore, the continued collection of
other application metrics has been pointed out as one of the next steps in
adopting DevOps at TCU.

\textbf{Regulatory Issues}:
Here, the group understood that, although the model has made it possible to
understand that the most important thing is to foster the \cc, many professionals
still thinking in a legalistic perspective, and the internal regulations about
the TCU' organizational structure establish that the responsibilities for issues
related to the application infrastructure are from its operations team,
which makes it difficult to consolidate a sense of shared responsibility.

There was no consensus about the best solution to resolve the constraints
contained in organizational structuring regulations. Some (P1 and P4) understand
that it would be appropriate part of the operations team to be transferred to
development sector. Others (P2 and P3) have demonstrated the understanding that
a change in the regulations is enough to define that there is shared
responsibility for issues related to application infrastructure. There is a
working group constituted in order to propose modifications in the regulations
to adjust these assignments to the DevOps scenario.

\textbf{Physical Distancing of Teams}:
The last challenge discussed during the focus group was the existence of
separate rooms for the development and operations teams. Physical distance has
been put as a factor that hinders the communication and the developing of
the \cc. The participants agreed that the physical approximation of the teams
involves questions related to restructuring regulations, as discussed above. If
the operations team is incorporated into the development team, the approximation
is likely to occur, otherwise it is necessary to seek another viable solution.
}


\subsubsection{General Considerations}

{
\color{blue}
Even though it was not the only point of debate, considerations about the model
permeated all the discussed topics, practical actions were highlighted that only
materialized due to the exchange of experiences that occurred during this
research. Many of the concepts presented in the model were visualized during
focus group discussions, this is not a mere coincidence and emphasizes that our
model is guiding, in a high level of abstraction, the actions of the TCU toward
DevOps adoption. In addition, it was possible to note a previously non-existent
concern about the \cc, the participants frequently placed their actions as part
of the efforts to foster the \cc.
}
