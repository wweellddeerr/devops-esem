\section{Final Remarks} \label{sec:conclusion}

In this paper, grounded in data collected from successfully DevOps adoption
experiences, we present a theory on DevOps adoption, a model of how to adopt
DevOps according with this theory, and a case of applying it in practice.

We found that the DevOps adoption involves a very specific relationship between
seven categories: \cat{agility}, \cat{automation}, \cc, \cat{continuous
measurement}, \cat{quality assurance}, \cat{resilience}, \cat{sharing and transparency}.
The core category of DevOps adoption is the \cc. Some of the
identified categories (i.e., automation and sharing and transparency) propitiate
the foundation of a \cc. Other categories
(i.e., agility and resilience) are expected consequences of this formation.
Finally, two other categories (i.e., continuous measurement and quality
assurance) work as both foundations and consequences. We call the foundations
categories as ``DevOps enablers'', and the consequences categories as ``DevOps outcomes''.
Crucially, this model simplifies the understanding of the
complex set of elements that are part of DevOps adoption, enabling it to be
more direct and with lower risk of focusing on wrong things. 

We experimented with 
this model in real settings, improving the benefits of adopting DevOps 
within a government institution that faced many problems with the separation between the 
development and operations teams. 

%% <<<<<<< HEAD
%% =======
%% \emph{Threats to validity.} There is no objective way to measure whether an
%% DevOps adoption was successful. So when we say that we investigate successful
%% adoptions, we are relying on the subjective perception of practitioners. Grounded
%% Theory offers rigorous procedures for data analysis, however, research
%% bias is inevitable in qualitative research. Certainly other researchers would
%% form a different theory analyzing the same data, but we believe that the main perceptions
%% would be preserved.

%% GT studies typically do not claim to be definitive, the resulting theory should
%% be modifiable in other contexts \cite{hoda2012developing}. What this means is that we do not claim
%% this theory to be absolute or final. We welcome extensions to the theory based
%% on unseen aspects or finer details of the present categories or potential discovery
%% of new dimensions from future studies. Future work can focus on investigate contexts
%% where DevOps adoption was not well succeeded, aim to validate if our model could be
%% relevant in this scenario.
%% >>>>>>> 76f76d814051c713d35571634d050832f6157350
