\section{Final Remarks} \label{sec:conclusion}

In this paper, grounded in data collected from successfully DevOps adoption
experiences, we present a theory on DevOps adoption, a model of how to adopt
DevOps according to this theory, and a case of applying it in practice.

We found out that the DevOps adoption involves a very specific relationship between
seven categories: \cat{agility}, \cat{automation}, \cc, \cat{continuous
measurement}, \cat{quality assurance}, \cat{resilience}, \cat{sharing and transparency}.
The core category of DevOps adoption is the \cc. Some of the
identified categories (i.e., automation and sharing and transparency) propitiate
the foundation of a \cc. Other categories
(i.e., agility and resilience) are expected consequences of this formation.
Finally, two other categories (i.e., continuous measurement and quality
assurance) work as both foundations and consequences. We call the foundations
categories ``DevOps enablers'', and the consequences categories ``DevOps outcomes''.
Crucially, this model simplifies the understanding of the
complex set of elements that are part of DevOps adoption, enabling it to be
more direct and to offer a lower risk of focusing on wrong things.

We experimented with
this model in real settings, improving the benefits of adopting DevOps
within a government institution that faced many problems with the separation between the
development and operations teams.
