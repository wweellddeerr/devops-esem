\begin{abstract}

  \emph{Background.} DevOps is a set of practices and cultural values
  that aims to reduce the
  barriers between development and operations
  teams. Due to its increasing interest and imprecise
  definitions, existing research works have tried to
  characterize DevOps.

  \emph{Aims.} Nevertheless, little is
  known about the \emph{practitioners' understanding}
  about successful paths for DevOps adoption. The lack of such understanding
  might hinder developers interested in learning DevOps or institutions willing
  to adopt DevOps. Therefore, our goal is to present real scenarios of DevOps
  adoption, including a theory, a model and a case study.

  \emph{Method.} We used classic Grounded Theory to build a explanation about 15
  scenarios of successful DevOps adoption in in 15 companies from different
  domains and across 5 countries. We proposed a model (i.e., a workflow for
  DevOps adoption.) and evaluated it through
  a case study at a Brazilian Government institution. We used a focus group to
  collect the company perceptions about DevOps adoption, including the
  application of our model.

  \emph{Results.} This paper
  presents a model to improve both the understanding and guidance
  of DevOps adoption. The model increments the existing view of
  DevOps by explaining the role and motivation of each
  category (and their relationships) in the DevOps adoption process.
  We organize this model in terms of \emph{DevOps enabler categories} and
  \emph{DevOps outcome categories}. We provide evidence that
  \emph{collaboration} is the core DevOps concern, contrasting with an existing
  wisdom that implanting specific tools to \emph{automate building, deployment,
  and infrastructure provisioning and management} is enough to achieve DevOps.

  \emph{Conclusions.} Altogether, our results contribute to (a) generating
  an adequate understanding of DevOps, from the perspective
  of practitioners; and (b) assisting other institutions in the
  migration path towards DevOps adoption.
\end{abstract}

\begin{CCSXML}
<ccs2012>
<concept>
<concept_id>10011007.10011074.10011134</concept_id>
<concept_desc>Software and its engineering~Collaboration in software development</concept_desc>
<concept_significance>500</concept_significance>
</concept>
<concept>
<concept_id>10011007.10011074</concept_id>
<concept_desc>Software and its engineering~Software creation and management</concept_desc>
<concept_significance>300</concept_significance>
</concept>
</ccs2012>
\end{CCSXML}

\ccsdesc[500]{Software and its engineering~Collaboration in software development}
\ccsdesc[300]{Software and its engineering~Software creation and management}

\keywords{DevOps, Grounded Theory, Software Development, Software Operations.}

\maketitle
