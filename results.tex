\section{How to Adopt DevOps?}

At the introduction of the paper, we have made the question: it there a
suitable way to adopt DevOps? Here, we present one possible response, based
on the analyses performed as detailed in the previous section. The highlighted
words represents categories and more detail about them is presented in the next
sections. The main point which should be sought is the formation of a
\textbf{collaboration culture} between the teams and activities of operations
related to software development. The other categories, many of which are also
present in other studies that have investigated DevOps, only make sense if
the practices related to them generate some increase in the level of
collaboration culture (how to measurement the level?). If all is automated:
deployment, infrascruture providing, monitoring, etc, but this automation is
maintened into a silo, where only one people or one team is able or responsible
to understand, adapt or evolve this automation, there is no increase in
collaboration level and, consequently, the DevOps adoption has not advanced.
The same is valid to the other categories. If transparency, sharing, etc don't
contribute to the collaboration culture, they don't contribute to DevOps
adoption.

"Estava faltando esse meio de campo na ligacao entre ops e devs (...) essas
ferramentas que a gente utilizou foram justamente para fazer a ligação do
DevOps" P3

In this section we present the results of the research.

\subsection{The Core Category: Collaboration Culture}

\section{DevOps Enablers}

\subsection{Automation}

\subsection{Continuous Measurement}

\subsection{Transparency}

\subsection{Sharing}

\section{DevOps Outcomes}

\subsection{Agility}

\subsection{Resilience}

\subsection{Quality Assurance}
