\section{Results}
In this section we present the results of the research.

\subsection{How to Adopt DevOps?}
At the introduction of the paper, we have made the question: it there a
suitable way to adopt DevOps? Here, we present one possible response, based
on the analyses performed as detailed in the previous section. The highlighted
words represents categories and more detail about them is presented in the next
sections. The main point which should be sought is the formation of a
\textbf{collaboration culture} between the teams and activities of operations
related to software development. The other categories, many of which are also
present in other studies that have investigated DevOps, only make sense if
the practices related to them generate some increase in the level of
collaboration culture (how to measurement the level?). If all is automated:
deployment, infrastructure providing, monitoring, etc, but this automation is
maintened into a silo, where only one people or one team is able or responsible
to understand, adapt or evolve this automation, there is no increase in
collaboration level and, consequently, the DevOps adoption has not advanced.
The same is valid to the other categories. If transparency, sharing, etc don't
contribute to the collaboration culture, they don't contribute to DevOps
adoption.

"Estava faltando esse meio de campo na ligacao entre ops e devs (...) essas
ferramentas que a gente utilizou foram justamente para fazer a ligação do
DevOps" P3

Given that the development of the collaboration culture should be the main
point of DevOps adoption, there is still the guidance of how to increase the
collaboration culture level. In the data analysis we conclude that some of
the emerged categories are part of what we call \textbf{DevOps enablers}.

\subsection{DevOps Enablers}

DevOps enablers are exactly the means commonly used to increase the level of
this collaboration culture in a DevOps adoption process. We have identified
four categories of DevOps enablers:

\begin{itemize}
\item Enabler 1 (E1): Automation;
\item Enabler 2 (E2): Continuous Measurement;
\item Enabler 3 (E3): Sharing;
\item Enabler 4 (E4): Transparency.
\end{itemize}

The enablers were listed in alphabetical order to reinforce that there is no
precedence order between them. There is no a enabler that should have
precedence in a DevOps adoption process. We have perceived that the adoption
process can follow a path that prioritizes specific enablers, on the condition
that the level of collaboration culture increases. For example, in 14
interviews \textbf{automation} was cited as a very important enabler to adopt
DevOps. But, one respondent answered the following:

"Embora a gente atualmente use automacao em um numero razoavel de cenarios,
nos conseguimos desenvolver bastante a nossa cultura sem uso de automacao e
eu penso que pode-se adotar DevOps com pouco ou ate mesmo nada de automacao" P8

That is, although automation is a very commonly used enabler, it is possible to
increase the level of collaboration culture without focus on automating. And
this premise is valid to the other enablers. Each enabler will be detailed in
the sequence of the paper.

\subsection{DevOps Outcomes}
We find also some categories of concepts, cited as part of DevOps adoption in
the companies, that doesn't produce the expected effect of a DevOps enabler.
So, we have perceived that this commonly categories are considered as expected
or required \textit{DevOps Outcomes}. We have identified four categories of
DevOps outcomes:

\begin{itemize}
\item Outcome 1 (E1): Agility;
\item Outcome 2 (E2): Quality Assurance;
\item Outcome 3 (E3): Resilience;
\end{itemize}

One well succeeded DevOps adoption typically increases the potential of agility,
resilience and quality assurance of a team. But, in some cases this potential
is not completely used due business decisions. For example, one respondent has
cited that at a first moment the company don't allowed the continuous
deployment of applications in production:

"Nos tinhamos condicoes e seguranca para publicar continuamente em ambiente de
producao, porem, por os gerentes tinham medo e decidiram por publicar apenas
semanalmente em producao" P9

In this case, the DevOps adoption increased the level of agility of the company,
but, this potential was not totally tapped.

As well as enablers, each outcome will be detailed in the sequence of the paper.

\subsection{The Core Category: Collaboration Culture}

\subsection{E1: Automation}

\subsection{E2: Continuous Measurement}

\subsection{E3: Sharing}

\subsection{E4: Transparency}

\subsection{O1: Agility}

\subsection{O2: Quality Assurance}

\subsection{O3: Resilience}
