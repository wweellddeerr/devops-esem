\section{A Theory on DevOps Adoption} \label{sec:results}

The results of a grounded theory study, as the name of the method itself
suggests, are grounded on the collected data, so the hypotheses emerge from
data. A grounded theory should describe the key relationships between the
categories that compose it, i.e., a set of inter-related hypotheses~\cite{hoda2017becoming}.
We present the categories of our grounded theory
about DevOps adoption as a network of the three categories of enablers (\cat{automation},
\cat{sharing and transparency}) that are commonly used to develop the core category
\cc, as discussed in the previous section. Based on our understanding,
implementing the enablers to develop the \cc typically leads
to concepts related to two categories of expected outcomes:
\cat{agility} and \cat{resilience}. Moreover, there are two categories that can be considered
both as enablers and as outcomes: \cat{continuous measurement} and \cat{quality assurance}.
In this section we describe the relationships between those categories, building a theory
of DevOps adoption.

\subsection{A General Path for DevOps Adoption}

In Section~\ref{sec:introduction} we presented the general question of this
research: is there any recommended path to adopt DevOps? Here, we elaborate a response,
based on the analyses conducted as detailed in Section~\ref{sec:research_method}. The main
point which should be formulated is the construction of a \cat{collaborative
culture} between the software development and operations teams and
related activities. According to our findings, the other categories,
many of which are also present in other studies that have investigated DevOps,
only make sense if the practices and
concepts related to them either contribute to the level of a \cc or lead to the expected consequences
of a \cc. This understanding induces several hypothesis, as discussed in
what follows.

\begin{mh}
\textbf{Hypothesis 1:} \textit{There is a group of categories related to DevOps adoption
that only make sense if used to increase the} \cc \emph{level. We
call this group of categories of \textbf{enablers}}.
\end{mh}

Based on this first hypothesis, the maturity of DevOps adoption does not
advance in situations where only one team is responsible to understand, adapt, or
evolve automation---even when such automation supports different activities like deployment, infrastructure provisioning,
monitoring. The same holds for the other \emph{enabling} categories. That is, in the situations which
\cat{transparency and sharing} do not contribute to
the \cc, they do not contribute to DevOps adoption as a whole. Some examples
that supports our first hypothesis include:

%\begin{mq}
%``\emph{Look, inside the operations sector there was some degree of automation. The guy
%had stored in his own machine bash scripts that helped him when setting up a
%server or when creating a new database instance. Nevertheless, there was no DevOps
%because there was no intrinsic relationship of this automation to the
%development process}" (P11, DevOps Supervisor, Brazil)
%\end{mq}


\begin{mq}
``\emph{DevOps involves tooling, but DevOps is not tooling. That is, people often
focus on using tools that are called `DevOps tools', believing that DevOps is
this. I always insist that DevOps is not tooling, DevOps involves the use of
tools properly, to improve software development procedures.}" (P2, DevOps
Consultant, Brazil)
\end{mq}


%% \begin{mq}
%% ``\emph{Keeping the culture alive remains a challenge to us, and it is very
%% important. Here in our company, for example, we have Tech Talks that are
%% monthly conversations that we have with the teams. The purpose of these Tech
%% Talks is to share knowledge about technologies and work processes increasing the
%% transparency of how everything works. We also have a Slack channel called
%% DevOps as Culture where we discuss things of DevOps culture. The idea is not to
%% let the culture die, we are always feeding it with something, because that is
%% the DevOps essence for us.}" (P12, Cloud Engineer, United States)
%% \end{mq}

\begin{mh}
\textbf{Hypothesis 2:} \textit{There is a group of categories related to DevOps adoption
that does not contribute to increase the} \cc \emph{level, but that instead are
pointed out as DevOps adoption related, because they emerge as an expected or
necessary consequence of the adoption. These categories represent the group of
\textbf{outcomes}}.
\end{mh}

In a first moment, the simple fact that a team is more
\cat{agile} in delivering software, or more \cat{resilient} in failure recovery, does not
contribute directly to bring operations teams closer to development teams.
Nevertheless, a signal of a mature DevOps adoption is an increasing of the capacity for continuously
delivering software (and thus being more \cat{agile})
and for building \cat{resilient} infrastructures.

\begin{mh}
\textbf{Hypothesis 3:} \textit{The categories \cat{Continuous Measurement} and \cat{Quality Assurance}
are both related to DevOps enabling capacity and to DevOps outcomes}.
\end{mh}

Measurement is cited as a typical responsibility of the operations team.
At the same time that sharing this responsibility reduces silos,
it is also cited that measurement is a necessary consequence of DevOps adoption. Particularly because
the continuous delivery of software requires more control,
which is supplied by concepts related to the \cat{continuous measurement} category.
The same premise is valid to the \cat{quality assurance} category. At first glance,
\cat{quality assurance} appears as one response to the context of agility in operations
as a result of DevOps adoption. But, the efforts in quality assurance of software products
increase the confidence between the development and operations teams, increasing the level
of \cc.

% \subsection{DevOps Enablers}

Altogether, DevOps enablers are the means commonly used to increase the level of
the \cc in a DevOps adoption process.
We have identified five categories of DevOps enablers:
\cat{ Automation}, \cat{Continuous Measurement}, \cat{Quality Assurance},
\cat{Sharing}, and {\cat{Transparency}. Another finding of our
study leads to our fourth hypothesis.

\begin{mh}
\textbf{Hypothesis 4:} \textit{There is no precedence between enablers in a DevOps adoption process}.
\end{mh}

We have realized that the adoption process might not have
to priorize any enabler, and a company that aims to implement
DevOps should start with  the enablers that seem more appropriate (in terms
of its specificities). Accordingly, we did not find any evidence that an enbler
is more efficient than another for creating a \cc. \cat{Automation} is the category
that appears more frequently in our study, though several participants make
clear that associating DevOps with automation is a misconception.
%% For
%% instance, although 14 respondents cite \cat{automation} as an important
%% enabler to adopt DevOps, some respondents also ponder that considering
%% automation with greater importance than other parts can actually be a risk:

\begin{mq}
``\emph{I think that the expansion of collaboration between teams involved other
things, it was not just automation. There must be an alignment with the
business needs. (...) I think that DevOps made possible a broader understanding
of software production and we were realizing exactly that it is not about
automating everything. (...) So, I see with caution a supposed vision that automate things can
be the way to implement DevOps.}" (P7, Support Analyst, Brazil)
\end{mq}

%\begin{mq}
%``\emph{Despite of we actually use automation in a reasonable number of scenarios,
%we have been able to develop our culture significantly without automation and I think that you can reach a
%good DevOps level with little or even no automation.}" (P8, DevOps Engineer, Brazil)
%\end{mq}

%% That is, although \cat{automation} is a very commonly used enabler, it is possible to
%% increase the level of \cc without focus on automating. And
%% this premise is valid to the other enablers.

% \subsection{DevOps Outcomes}

DevOps outcomes is that group of categories that does not produces primarily the
expected effect of a DevOps enabler, typically concepts that are expected as
consequences of an adoption of DevOps. We have identified four categories that
can work as DevOps outcomes: \cat{agility}, \cat{continuous measurement},
\cat{quality assurance}, and \cat{software resilience}. Note that,
as mentioned before, \cat{continuous measurement} and \cat{quality assurance}
are both enablers and outcomes.

That is, a well succeeded DevOps adoption typically increases the potential of
\cat{agility} of teams and enables \cat{continuous measurement}, \cat{quality assurance} and
\cat{resilience} of applications.
However, in some situations, this potential is not completely used due business
decisions. For example, one respondent cited that, at a first moment, the
company did not allowed the continuous deployment (more potential of agility)
of applications in production:

\begin{mq}
``\emph{We had conditions and security to continuously publish in production,
however, in the beginning the managers were afraid and decided that the
publication would happen weekly.}" (P9, IT Manager, Brazil)
\end{mq}
