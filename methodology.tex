\section{Research Method} \label{sec:research_method}

We adopted Grounded Theory (GT) as the research method. GT is a method
originally proposed by Glaser and Strauss~\cite{glase1967discovery}, which has
as distinguishing features (1) the absence of clear research hypothesis upfront
and (2) limited of literature exposure at the beginning of the research. GT
is a theory-development approach (the hypothesis emerge as the results of
a research investigation), in contrast with more traditional
theory-testing approaches~\cite{coleman2007using}---for instance,  those that
use statistical methods to either confirm or refute preestablished hipothesis. 
We used GT as the research method due to the following reasons:

%Some of related work claimed to used ``GT inspired'' approaches\gnote{REF}. It is a very
%common rhetoric in recent research on software engineering, but only research
%that embodies GT’s core principles should claim to be a grounded theory study
%\cite{stol2016grounded}.\gnote{eu removi esse paragrafo pq acho que ele facilmente compraria briga com muitos revisores}


\begin{itemize}

\item GT is a consolidated method in other areas of research, like sociology\gnote{REF},
nursing\gnote{REF}, education\gnote{REF}, and finances\gnote{REF}. GT is also being increasingly employed
to study software engineering topics~\cite{Hoda:2017:ICSE,stol2016grounded,Waterman:2015:ICSE};

\item GT is considered an adequate approach to investigate scenarios that present 
  questions that aims to characterize scenarios under a personal perspective of those
  engaged in a discipline or activite~\cite{barnsteiner2002using},
which is exactly the scenario here: what are the successfull adoption paths for DevOps?

\item GT allows researchers to build an independent and original understanding, 
which is adequate to collect empirical evidence directly from the
practice on industry\gnote{nao entendi bem a relação do primeiro item com o segundo?}. The evidence
is only reintegrated back with the existing literature after the step of
theory construction.

\end{itemize}

Since the publication of the Glaser and Strauss original version of GT~\cite{glase1967discovery},
several modifications and variations have been proposed to the method, coming to
exist at least seven different versions of Grounded Theory~\cite{denzin2007grounded}.
The main versions are those of Glaser, Strauss and
Charmaz~\cite{stol2016grounded}. The study of Stol and colleagues~\cite{stol2016grounded}
explore the main aspects of GT versions and recommend authors of GT studies to
specify which version of the method the study is built upon. Here we chose the classic
Glaser and Strauss version, mainly because we did not have a research
question at the beginning of our research, exactly as suggested in the classic
version. We actually started from an area of interest: successfully DevOps adoption
in industry. In addition, research works in
software engineering predominantly use this version~\cite{stol2016grounded}.

% \subsection{GT Procedures}

% In Figure~\ref{fig1}, reproduced from the study of Adolph and colleagues~\cite{adolph2011using}, shows the procedure of the grounded theory research methods followed in the conduction of this
% research.

% \begin{figure}[htpb]
%   \centering
%   \includegraphics[width=0.5\textwidth,natwidth=821,natheight=617]{GT.png}
%   \caption{The GT Method (reproduced from the study of Adolph and colleagues~\cite{adolph2011using}).}
%   \label{fig1}
% \end{figure}

% This figure depicts four main steps (enumerated as A -- D).

We carried out our research using an existing 
guideline about how to conduct a research based on 
Grounded Theory~\cite{adolph2011using}. This guideline organizes 
a GT investigation in three steps: \emph{Open Ccoding} Data Collection,  
\emph{Selective Coding} Data Analysis using, and Theoretical Coding. 
 

\gnote{acho que antes disso, é preciso dizer como você encontrou e abordou
os possiveis entrevistados? Jogou um convite nas redes socias? foi por
conveniencia (p.e., vc conhecia alguem?), etc}

\begin{enumerate}[label=(\Alph*)]
\item {\bf Open Ccoding Data Collection.} We started our research 
  by collecting and analysing data from companies that have successfully adopted DevOps. 
  To this end, we conducted a \emph{raw data analysis} that search for patterns of 
  incidents to indicate concepts,  and then grouped these concepts into 
  categories~\cite{stol2016grounded}.

\item {\bf Selective Coding Data Analysis.} In the second step we evolve 
  the initial set of 
  categories by comparing new incidents with the previous ones. Here the goal is 
  to identify a ``core category''~\cite{stol2016grounded}.
  The core category is responsible for enabling the integration of the other
  categories and structuring the results into a dense and consolidated grounded
  theory~\cite{jantunen2014using}. The identification of the core category
  represents the end of the open-coding phase and the beginning of the selective coding.
  In the selective coding, we only consider the specific variables that are directly 
  related to the core category, in order to enable the production of an harmonic
  theory~\cite{coleman2007using,hoda2011impact}. Selective coding ends when we 
  achieve a theoretical saturation is achieved, {\color{red}which occurs when new data (............).}

\item {\bf Theoretical Coding.} After saturation, we built a theory that explains 
 the categories and the relationships between the categories. Additionally, we reintegrated 
 our theory with the existing literature, {\color{red}which allowed us to compare our proposal 
 with other theories about DevOps}. That is, using a Grounded Theory approach, 
 one should only conduct a literature review in later stages of a research,  
in order to avoid external influences to conceive a theory~\cite{adolph2012reconciling}.

\end{enumerate}

Throughout the process, we wrote memos capturing thoughts and analytic
processes; the memos support the emerging concepts, categories, and their
relationships~\cite{adolph2012reconciling}.

\gnote{para cada um dos itens acima, poderiamos colocar exemplos reais
do trabalho? p.e., citar 1-2 memos?}

% The following sub-sections present details of procedures applied in this study,
% containing some examples to illustrate their application.

% \subsection{Data Collection}

Regarding data collection, we conducted semi-structured interviews with 15 practitioners of companies from
Brazil, Ireland, Portugal, Spain, and United States. This practitioners claim
to have contributed to DevOps adoption processes in their companies. Participants
were recruited using two approaches: (1) through direct contact in a \emph{DevOps Days} 
event and (2) through  general
calls for participation posted on DevOps user groups, social networking sites,
and local communities. A variety of company types were covered, in order to
achieve a heterogeneous perspective and increase the potential of generalization
of the results. Table~\ref{participant_table} presents the participant
characteristics. 
To maintain anonymity, in conformance with the human ethics guidelines,
hereafter we will refer to the participant as P1--P15 (first column).
The second column shows the role of
each participant in her respective company. The remaining columns list their
software development experience in years (SWX), DevOps experience in years (DX),
country of work (CN), main domain of the company, and company size (CS).


\begin{table}[t]
\centering
\caption{Participant Profile. SWX means software development experience in years, DX means DevOps experience in years, CN means country of work, and CS means company size (CS).}
\label{participant_table}
\begin{tabular}{ccccccc}
\textbf{P\#}          & \textbf{Role}         & \textbf{SWX} & \textbf{DX} & \textbf{CN}   & \textbf{Domain}    & \multicolumn{1}{l}{\textbf{CS}} \\
P1                   & DevOps Developer      & 9            & 2           & IR            & IT                 & S                               \\

P2                   & DevOps Consultant       & 9            & 3           & BR            & IT                 & M                               \\

P3                   & DevOps Developer      & 8            & 1           & IR            & IT                 & S                               \\

P4                   & Computer Tech.        & 10           & 2           & BR            & Health             & S                               \\

P5                   & Systems Engineer      & 10           & 3           & SP            & Telecom            & XL                              \\

P6                   & Developer             & 3            & 1           & PO            & IT                 & S                               \\

P7                   & Support Analyst       & 15           & 2           & BR            & Telecom            & L                               \\

P8                   & DevOps Engineer       & 20           & 9           & BR            & Marketing              & M                               \\

P9                   & IT Manager            & 14           & 8           & BR            & IT                 & M                               \\

P10                  & Network Admin.        & 15           & 3           & BR            & IT                 & S                               \\

P11                  & DevOps Supervisor                & 6            & 4           & BR            & IT                  & M                               \\

P12                  & Cloud Engineer              & 9            & 3           & US            & IT                  & L                               \\

P13                  & Technology Manager                 & 18            & 6           & BR            & Food                  & M                               \\

P14                  & IT Manager            & 7            & 2           & BR            & IT                  & S                               \\

P15                  & Developer        & 3            & 2           & BR            & IT                  & S
\end{tabular}
\end{table}



The interview were conducted over one year using Skype calls.
Data collection and analysis were iterative so the collected data helped guide
future interviews. {\color{red}The questions of interview evolved according to the progress
of the research...} 

With respect to \emph{data analysis}, the interviews were voice recorded, transcribed, and analysed. The first moment
of the analysis, called open coding in GT, started immediately after the
transcription of the first interview, which 
was used to evolve the interview script to be used in
the second interview, and so on. The open coding lasted until there was no
doubt about the core category of the study. Similar to what is described in
the work of Adolph and colleagues\cite{adolph2012reconciling}, we started
with a strong candidate core category yet not consolidated. Until
the analysis of the fourth interview, we cogitated \emph{automation} as core
category because it is a recurring pattern in our data. However, we quickly
realized that the ``automation" category did not explain most of the behaviors
or events in our data. At the same time, we started to understand that 
\emph{collaboration culture} also appeared recurrently in the analysis and with more
potential to explain the remaining events. We then started to ask explicitly
about the role of the automation and how the collaboration culture is formed
in a DevOps adoption process.

After the adaptations made in the script and analysis of new data in a constant
comparison process, taking into account the previous analyses and the
respective memos written during all the process, after the analysis of the tenth
interview, we concluded that ``collaboration culture" was unequivocal the core
category regarding how DevOps was successfully adopted in the studied companies.
At this moment, the open coded ended and the selective coding started.

After the discovery of the core category, we started to restrict the coding only
to specific variables that were directly related to the core category and their
relationships: the selective coding.
With more three interviews and respective analysis, we realized that
the new data added less and less content to the emerging theory. Thtat is, the
explanation around how the collaboration culture is developed providing
DevOps adoption showed signs of saturation. We then conducted more two
interviews to conclude that we had reached the theoretical saturation.

After saturation, we started the theoretical coding to find a way to integrate
all the concepts, categories and memos in the form of a cohesive and
homogeneous theory, where we have pointed out the role of the categories as
enablers and outcomes, as shown above. We present more details about 
the results of our theoretical coding phase in the next section. 


To illustrate the coding procedures, we present an example of working from
interview transcript to the findings for one of the categories: automation.

\gnote{seria bom linkar aqui os itens A--B descritos na seção anterior.}

Apresentar aqui a exemplificacao da codificacao....

\subsection{Reintegrate with Literature (?)}
