\section{Threats to Validity}\label{sec:threats}

Regarding construct validity,  we are actually relying on the subjective
practitioners' perception when we stated that we performed our study considering successful cases
of DevOps adoptions. However, currently, there is no objective way to measure whether or not a
DevOps adoption was successful.
Although Grounded Theory offers rigorous procedures for data analysis, our
qualitative research may contain some degree of research bias. Certainly, other
researchers might form a different interpretation and theory after analyzing
the same data, but we believe that the main perceptions would be preserved.
This is a typical threat related to GT studies, which do not claim to generate
definitive findings. The resulting theory, for instance, might
be different in other contexts \cite{hoda2012developing}.

For this reason, we do not claim
that our theory is absolute or final. We welcome extensions to the theory based
on unseen aspects or finer details of the present categories or potential discovery
of new dimensions from future studies.
Future work can also focus on investigating contexts
where DevOps adoption did not succeed, aiming to validate if our model could be
relevant in this scenario too. Finally, regarding external validity, although we
considered in our study the point of view of practitioners with different
backgrounds, working in companies from different domains, and distributed across
five countries, we do not claim that our results are valid for
other scenarios---although we almost achieved saturation
after the 12$^{th}$ interview. Accordingly, our degree of heterogeneity complement
previous studies that mostly focus in a single company (as we will discuss next).

The focus group was moderated by one of the researchers, the participants were
arbitrarily invited, without a general call, and they are co-workers of one of
the researchers. Although they were chosen arbitrarily, the choice was made
precisely by the prior knowledge of which professionals were directly involved
in DevOps adoption at TCU. To mitigate this threat, the participants were
informed the purpose of the group was to obtain an evaluation of the DevOps
adoption as a whole and that they had total freedom to expose their real
opinions, whether they were favorable or not to the implanted model.
