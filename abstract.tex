\begin{abstract}
  DevOps is a set of practices that has gained increasing
  attention in the industry and that aims to reduce the
  barriers between development and production
  teams. Due to its increasing interest and imprecise
  definitions, existing research works have tried to
  characterize DevOps, mainly using a set of concepts and related practices.
  However, little is known about the \emph{practitioners understanding}
  about successful paths for DevOps adoption, which might
  hinder institutions to start the use DevOps practices. 
  Based on a Grounded Theory study, considering 15 well succeeded DevOps
  adoptions in companies across five countries, we
  present a model to improve both the understanding and guidance
  of DevOps adoption. The model increments the existing view of
  DevOps by explaining the role and motivation of each
  element (and their relationships) in the DevOps adoption process.
  We organize this model in terms of \emph{DevOps enabler categories} and
  \emph{DevOps outcome categories}, and explain that \emph{collaboration}
  is the key aspect of DevOps, contrasting with an existing wisdom that
  \emph{automating building and deployment processes} is enough to
  achieve DevOps. We have been using this understanding to increase
  the maturity level of DevOps adoption in the Brazilian Federal Court of Accounts,
  a Brazilian Federal Government institution. In this paper
  we also report this experience. Altogether, our results
  contribute to (a) generating an adequate understanding of DevOps,
  from the perspective of
  practitioners; and (b) assisting other institutions in the
  migration path towards DevOps. 
  \gnote{faltou uma breve descrição dos achados. Qual o key take-away?}
\end{abstract}

% no keywords
