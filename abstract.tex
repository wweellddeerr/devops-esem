\begin{abstract}

  \emph{Background.} DevOps is a set of practices and cultural values
  that aims to reduce the
  barriers between development and operations
  teams. Due to its increasing interest and imprecise
  definitions, existing research works have tried to
  characterize DevOps---mainly using a set of concepts and related practices.

  \emph{Aims.} Nevertheless, little is
  known about the \emph{practitioners understanding}
  about successful paths for DevOps adoption. The lack of such understanding
  might hinder institutions to adopt DevOps practices. Therefore, our goal
  here is to present a theory about DevOps adoption, highlighting the
  main related concepts that contribute to its adoption in industry.

  \emph{Method.} Our work builds upon the classical Grounded
  Theory approach. We interviewed practitioners
  that contributed to the adoption of DevOps in 15 companies from different
  domains and across five countries. We empirically evaluate our model through
  a case study, whose goal is to increase the maturity level of
  DevOps adoption at the Brazilian Federal Court of Accounts,
  a Brazilian Federal Government institution.

  \emph{Results.} This paper
  presents a model to improve both the understanding and guidance
  of DevOps adoption. The model increments the existing view of
  DevOps by explaining the role and motivation of each
  category (and their relationships) in the DevOps adoption process.
  We organize this model in terms of \emph{DevOps enabler categories} and
  \emph{DevOps outcome categories}. We provide evidence that
  \emph{collaboration} is the core DevOps concern, contrasting with an existing
  wisdom that implanting specific tools to \emph{automate building, deployment,
  and infrastructure provisioning and management} is enough to achieve DevOps.

  \emph{Conclusions.} Altogether, our results contribute to (a) generating
  an adequate understanding of DevOps, from the perspective
  of practitioners; and (b) assisting other institutions in the
  migration path towards DevOps adoption.
\end{abstract}

\begin{CCSXML}
<ccs2012>
 <concept>
  <concept_id>10010520.10010553.10010562</concept_id>
  <concept_desc>Computer systems organization~Embedded systems</concept_desc>
  <concept_significance>500</concept_significance>
 </concept>
 <concept>
  <concept_id>10010520.10010575.10010755</concept_id>
  <concept_desc>Computer systems organization~Redundancy</concept_desc>
  <concept_significance>300</concept_significance>
 </concept>
 <concept>
  <concept_id>10010520.10010553.10010554</concept_id>
  <concept_desc>Computer systems organization~Robotics</concept_desc>
  <concept_significance>100</concept_significance>
 </concept>
 <concept>
  <concept_id>10003033.10003083.10003095</concept_id>
  <concept_desc>Networks~Network reliability</concept_desc>
  <concept_significance>100</concept_significance>
 </concept>
</ccs2012>
\end{CCSXML}

\begin{CCSXML}
<ccs2012>
<concept>
<concept_id>10011007.10011074.10011134.10003559</concept_id>
<concept_desc>Software and its engineering~Open source model</concept_desc>
<concept_significance>500</concept_significance>
</concept>
<concept>
<concept_id>10003120.10003130</concept_id>
<concept_desc>Human-centered computing~Collaborative and social computing</concept_desc>
<concept_significance>300</concept_significance>
</concept>
</ccs2012>
\end{CCSXML}

\ccsdesc[500]{Software and its engineering~Open source model}
\ccsdesc[300]{Human-centered computing~Collaborative and social computing}

\keywords{DevOps, Grounded Theory, Software Development, Software Operations.}

\maketitle
