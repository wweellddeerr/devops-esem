\begin{abstract}
DevOps is a term that has emerged in the industry. Using predominantly
literature reviews, academic studies have sought to characterize it as a set
of concepts with related practices. This view of DevOps does not provides clear
guidance to companies that want to adopt DevOps, changing your software
development paradigm. Empirical studies on industry, where the term DevOps was
born, seem to be an appropriate approach to investigate DevOps adoption with
more clear guidance to new practitioners. Based on a Grounded Theory study of
15 well succeeded DevOps adoptions in companies across five countries, we
present a model to improve the understanding and guidance of DevOps adoption.
The model increments the existing view of DevOps providing explanation about
the role and motivation of each element in a DevOps adoption process and the
relationship between them. The model was applied in the DevOps adoption of
Brazilian Federal Court of Accounts, a company of Brazilian Federal Government.
\end{abstract}

% no keywords
