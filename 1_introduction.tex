\section{Introduction} \label{sec:introduction}

DevOps is a set of practices and cultural values that has emerged in the
software development industry~\cite{devops_a_definition_xp_15,dimensions_of_devops_xp_15,extending_dimensions_icsea_16,characterizing_devops_sbes_2016}. Even before
the existence of the term --- a mix of ``development'' and ``operations''
words~\cite{httermann2012devops} --- companies like Flickr~\cite{flickr}
had already pointed out the need to break the existing separation between
the operations and software development teams. Since then, the term
that although appeared without a clear delimitation, gained strengths and interests
from companies that perceived the benefits of applying agile practices in
\emph{operation tasks}.
DevOps claimed benefits include increased organizational IT
performance and productivity, cost reduction in software lifecycle, improvement
in operational efficacy and efficiency, better quality of software products, and
greater business alignment between development and operations
teams~\cite{characterizing_devops_sbes_2016,state_of_devops,DevOps_Adoption_Benefits_and_Challenges}.
However, the adoption of DevOps is still a challenging task. Even though there is a
plethora of information, practices, and tools related to DevOps, it is still unclear
how one could leverage such rich, yet scattered, information in an organized and
structured way to properly adopt DevOps.

Existing research works have proposed a
number of DevOps characterizations, for instance, as a set of concepts with
related
practices~\cite{devops_a_definition_xp_15,dimensions_of_devops_xp_15,extending_dimensions_icsea_16,characterizing_devops_sbes_2016,cooperation_dev_ops_esem_2014,qualitative_devops_journalsw_17}. Although some
of these studies leverage qualitative approaches to gather practitioners' perception (for instance,
conducting interviews with them), they focus on characterizing DevOps,
instead of providing recommendations to assist on DevOps adoption. Consequently,
our {\bf research problem} is that the obtained DevOps characterization provides a
comprehensive understanding of the elements that constitute DevOps, but do not
provide detailed guidance to support newcomers interested in adopting DevOps.
As a consequence, many practical and timely questions still remain open, for
instance: (1) Is there any recommended path to adopt DevOps? (2) Since
DevOps is composed by multiple elements~\cite{dimensions_of_devops_xp_15}, do
these elements have the same relevance, when adopting DevOps?
(3) What is the role played by elements such as measurement, sharing, and automation
in a DevOps adoption? To answer these questions, we need a holistic
understanding of the paths followed in successful DevOps adoptions.

This paper is a continuation of a previous study~\cite{Luz:2018:ESEM}.
In our previous study, we introduced a model based on the perceptions of practitioners from
15 companies across five countries that successfully adopted DevOps. The model
was constructed based on a classic Grounded Theory (GT) approach,
and makes clear that practitioners interested in adopting DevOps should focus on building a
\cc, which prevents common pitfalls related to focusing on tooling or automation~\cite{Kromhout:2017:Queue}.
We instantiated our model in the Brazilian Federal Court of
Accounts (hereafter TCU),\footnote{http://www.tcu.gov.br/} a Brazilian Federal Government institution. TCU was
bogged down in implanting specific DevOps tools, repeating the same non-DevOps
problems, with conflicts between development and operations teams about how to
divide the responsibilities related to different facets in the intersection
between software development and software provisioning.

When instantiated,
our model helped TCU to change its focus to improve the collaboration between teams, and to use the tooling
to support (rather than being the goal of) the entire process.
In our previous work
we briefly introduced this experience at TCU.
Complementing this initial research, in this paper we report the current status
of the DevOps adoption at TCU as a whole, including an empirical assessment of our model.
We collect the TCU perceptions through a focus group with four of its directly
involved professionals, two of each team (development and operations). Based on
the results of the focus group, we review our theory in this paper, which is
detailed using a well-known framework for building theories on software engineering~\cite{sjoberg2008}.
The main contributions of this paper are the following:

\begin{itemize}
\item A review and extended description of our work that builds a model and
  a theory about DevOps adoption. This theory might support practitioners interested in adopting DevOps,
  based on evidence acquired from industry peers.

\item An instantiation of this theory in a real world, non-trivial context. Specifically, TCU,
  as a government institution, is rather different from typical \emph{tech companies} that have
  successfully reported the adoption of DevOps, which substantiate the DevOps potential, even in more traditional
  companies.

\item The results of a focus group that evidenced that the use of our theory in TCU brought several
  benefits and now DevOps practices have been disseminated at TCU. We also report initial benefits
  of DevOps adoption in this company, in particular the introduction of an agile approach for software
  deployment and provisioning.

\end{itemize}

The rest of the work is organized as follows: Section~\ref{sec:method} presents our method
and the settings used to conduct this research. Section~\ref{sec:categories_concepts} describes our
preliminary findings on what constitute DevOps, along with the main enablers
and outcomes categories. In Section~\ref{sec:results} we present our theory regarding
a successful DevOps adoption, while Section~\ref{sec:case_study} describes our three step
model that one could use to adopt DevOps. Next, in Section~\ref{sec:TCU} we discuss
the findings of applying our model in a real world setting. Section~\ref{sec:threats}
discusses some threats to validity, while Section~\ref{sec:related_work} presents the related
work. Finally, Section~\ref{sec:conclusion} concludes our work and suggest eventual research
opportunity for future work.
