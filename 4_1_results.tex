\section{A Theory on DevOps Adoption} \label{sec:results}

The results of a grounded theory study, as the name of the method itself
suggests, are grounded on the collected data, so the hypotheses emerge from
data. A grounded theory should describe the key relationships between the
categories that compose it, i.e., a set of inter-related hypotheses~\cite{hoda2017becoming}.
We present the categories of our grounded theory
about DevOps adoption as a network of the two categories of enablers (\cat{automation},
\cat{sharing and transparency}) that are commonly used to develop the core category
\cc, as discussed in the previous section. Based on our understanding,
implementing the enablers to develop the \cc typically leads
to concepts related to two categories of expected outcomes:
\cat{agility} and \cat{resilience}. Moreover, there are two categories that can be considered
both as enablers and as outcomes: \cat{continuous measurement} and \cat{quality assurance}.
In this section, we describe the relationships between those categories, building a theory
of DevOps adoption.

\subsection{A General Path for DevOps Adoption}

In Section~\ref{sec:method} we presented the general questions of this
research, including: How do practitioners characterize a successfully path
   for DevOps adoption? Here, we elaborated a response to this question,
based on our grunded theory study. The main
point that should be formulated is the construction of a \cat{collaborative
culture} between the software development and operations teams and
related activities. According to our findings, the other categories,
many of which are also present in other studies that have investigated DevOps,
only make sense if the practices and concepts related to them either contribute
to the level of a \cc or lead to the expected consequences
of a \cc. This leads to a general hypothesis of our work, which is:

\begin{mh}
  {\bf General Hypothesis:} \textit{DevOps brings several benefits
  related to Agility, Resilience, and Software Quality
  Assurance. Nonetheless, to achieve this benefits,
  it is highly recommended to work towards a} \cc,
  \textit{removing silos and and implementing a more direct
  communication between the teams}. 
\end{mh}

The quote bellow makes explicit the relevance of a \cc
---as well as the challenges to keep this culture-to
achieve the benefits with DevOps adoption.

\begin{mq}
  ``\emph{Keeping (the collaborative) culture alive is still a challenge for us,
  and we consider it very important. Here in the company, for example, we have
  tech talks that are monthly conversations with the teams. The
  purpose of these Tech Talks is to share knowledge, technologies, and work
  procedures, by increasing the transparency about how everything works. We also
  have a DevOps culture Slack channel, where we discuss DevOps. The idea
  is not to let the culture die, \ldots, because that is the essence of
  DevOps for us}.'' (P12, Cloud Engineer, United States)
\end{mq}


This hypothesis emerged from our understanding about the
perceptions of the practitioners on the concepts that might
positively influence the adoption of DevOps. We built
this understanding iteratively, through discussions
considering the transcripts, memos, categories and their relationships. Surelly, during this
process, different opinions emerged, though in the end
the authors of this paper agreed with this general hypothesis.
For intance, in certain moments of the research, we used to believe that DevOps would
actually mean the \emph{end of the operations teams}. After
several rounds of discussion, we concluded that DevOps actually
focus on a collaborative approach for executing development and
operation tasks. This understanding induces four sub-hypothesis, as discussed in what
follows.

\begin{mh}
\textbf{Hypothesis 1:} \textit{Certain categories related to DevOps adoption
only make sense if used to increase the} \cc \emph{level. We
call this set of categories of \textbf{enablers}}.
\end{mh}

Based on this hypothesis, the maturity of DevOps adoption does not
advance in situations where only one team is responsible to understand, adapt, or
evolve automation---even when such automation supports different activities like
deployment, infrastructure provisioning and monitoring. The same holds for the
other \emph{enabling} categories. That is, in situations that
\cat{transparency and sharing} do not contribute to
the \cc, they do not contribute to DevOps adoption as a whole. This is
clear when one of the participants of our study states that

\begin{mq}
``\emph{DevOps involves tooling, but DevOps is not tooling. That is, people often
focus on using tools that are called `DevOps tools', believing that this is
what DevOps is. I always insist that DevOps is not tooling, DevOps involves the
proper user of tools to improve software development procedures.}'' (P2, DevOps
Consultant, Brazil)
\end{mq}

\begin{mh}
\textbf{Hypothesis 2:} \textit{Some other categories are not related to DevOps
adoption for contributing to increase the} \cc \emph{level, but insted for emerging
as an expected or necessary consequence of the adoption. These 
represent the set of \textbf{outcome} categories}.
\end{mh}

In a first moment, the simple fact that a team is more
\cat{agile} in delivering software, or more \cat{resilient} in failure recovery, does not
contribute directly to bringing operations teams closer to development teams.
Nevertheless, a signal of a mature DevOps adoption is an increasing capacity for continuously
delivering software (and thus being more \cat{agile})
and for building \cat{resilient} infrastructures. 

\begin{mh}
\textbf{Hypothesis 3:} \textit{The categories \cat{Continuous Measurement} and \cat{Quality Assurance}
  are both related to DevOps enabling capacity and to expected
  DevOps outcomes}.
\end{mh}

Measurement is cited as a typical responsibility of the operations team.
At the same time that sharing this responsibility reduces silos,
it is also cited that measurement is a necessary consequence of DevOps adoption. Particularly because
the continuous delivery of software requires more control,
which is supplied by concepts related to the \cat{continuous measurement} category.
The same premise is valid to the \cat{quality assurance} category. At first glance,
\cat{quality assurance} appears as one response to the context of agility in operations
as a result of DevOps adoption. But, the efforts in quality assurance of software products
increase the confidence between the development and operations teams, increasing the level
of \cc.

Altogether, DevOps enablers are the means commonly used to increase the level of
the \cc in a DevOps adoption process.
We have identified five categories of DevOps enablers:
\cat{ Automation}, \cat{Continuous Measurement}, \cat{Quality Assurance},
\cat{Sharing}, and {\cat{Transparency}. Another finding of our
study leads to our fourth hypothesis.

\begin{mh}
\textbf{Hypothesis 4:} \textit{There is no precedence between enablers in a DevOps adoption process}.
\end{mh}

We have realized that the adoption process might not have
to prioritize any enabler, and a company that aims to implement
DevOps should start with  the enablers that seem more appropriate (in terms
of its specificities). Accordingly, we did not find any evidence that an enabler
is more efficient than another for creating a \cc. \cat{Automation} is the category
that appears more frequently in our study, though several participants make
clear that associating DevOps with automation is a misconception.

%\begin{mq}
%``\emph{I think that the expansion of collaboration between teams involved other
%things. It was not just automation. There must be an alignment with the
%business needs. (...) I think that DevOps enabled a broader understanding
%of software production and we realized the very fact that it is not about
%automating everything. (...) So, I see with caution a supposed vision that
%automating things can be the way to implement DevOps.}'' (P7, Support Analyst, Brazil)
%\end{mq}

DevOps outcomes are the categories that does not primarily produce the
expected effect of an {\bf enabler}, typically concepts that are expected as
consequences of an adoption of DevOps. We have identified four categories that
can work as DevOps outcomes: \cat{agility}, \cat{continuous measurement},
\cat{quality assurance}, and \cat{software resilience}. Note that,
as mentioned before, \cat{continuous measurement} and \cat{quality assurance}
are both enablers and outcomes.

That is, a well succeeded DevOps adoption typically increases the potential of
\cat{agility} of teams and enables \cat{continuous measurement}, \cat{quality assurance} and
\cat{resilience} of applications.
However, in some situations, this potential is not completely used due to business
decisions. For example, one respondent cited that, at a first moment, the
company did not allow the continuous deployment (more potential of agility)
of applications in production.

%\begin{mq}
%``\emph{We had conditions and security to continuously publish in production,
%however, in the beginning, the managers were afraid and decided that the
%publication would happen weekly.}'' (P9, IT Manager, Brazil)
%\end{mq}

Considering the hypothesis we build from our understanding about
a successfully path for DevOps adoption, we answer our
first research question ``\emph{(RQ1) How do practitioners characterize a successfully path
for DevOps adoption?}''

\begin{mr}
  \textbf{(RQ1) Answer:} In order to successfully
  conduct an effort for DevOps adoption, practitioners
  suggest the focus on building a \cc (this is
  the essence of DevOps), which should be achieved through Automation,
  Transparency and Knowledge Sharing, Continuous Measurement,
  and Quality Assurance. Without setting up \cc as the main
  goal, the chances of failing to achieve the expected
  benefits of adopting DevOps (e.g., Agility and Resilience)
  increase. 
\end{mr}

