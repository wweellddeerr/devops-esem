\section{Categories and Concepts} \label{sec:categories_concepts}
In this section we present details and related concepts of each category of
DevOps adoption.

\subsection{The Core Category: Collaboration Culture}
In this section we present the details of what exactly means build and
increase a collaboration culture in DevOps adoption. The core category has six
related concepts and they are described in the next paragraphs.

The collaboration culture is essentially about remove the barriers that
constitutes the silos between development and operations teams and activities.
The operations tasks - like deployment, infrastructue provisioning and
management, monitoring - should be present in day to day of development. This
is the first concept related to collaboration culture category:
\textbf{operations in day to day development}.

``A very important step that we took was to bring the deployment into day to day
development, no more waiting for a specific day of the week or month. We wanted
to do deployment all the time, even if in a first moment it were not in
production, a staging environment was enough. What we wanted was to embed
deployment into development. Of course, to carry out the deployment
continuously, we had to provide all the necessary infrastructure at the same
pace" - P13, Technology Manager, Brazil

Without DevOps, a common scenario is an accelerated software development
without concerns about operations. At the end, when the development team has a
minimum viable software product, it is sent to the operations team for
publication. Knowing few things about the nature of the software and how it
was produced, the operations team has to create and configure an environment
and to publish the software. In this scenario, software delivery is typically
delayed and conflicts between teams manifest themselves. When a collaboration
culture is fomented, teams collaborate to perform the tasks from the first day
of software development. With the constant exercise of provisioning, management,
configuration and deployment, the software delivery becomes more natural,
reducing the delays and, consequently, the conflicts between teams.

``We work in a very agile way, but we had a 15-day sprint where we focused on
producing software, so we produced in a very absurd speed, and at the time of
devlivering what was produced, complications appeared. The job of setting up
the whole environment e make a manual deployment, this was not part of our
sprint, we focused only on coding the application. Then delivey was delayed,
having to deliver with delays of weeks, which was not good either for us or for
the clients" - P6, Developer, Portugal

The development team no longer need to stop their work waiting for the creation
of one application server, or for the execution of some database script, or for
the publication of a new version of the software in a staging environment.
Everyone needs to know the way this is done and, with the collaboration of the
operations team, this can be performed in day to day. If any task can be
performed by development team and there is trust between teams, this task is
incorporated into the development process in a natural way, manifesting the
second concept related to collaboration culture category: \textbf{software
development development}.

``We had several people working on development, the amount of developers is
impressive. It was not feasible so many developers generating artifacts and
stoping their work to wait another completely separate team to publish. Or
needing a test environment and having to wait for the operations team to
provide it when possible. These activities have to be available to quickly
serve the development team. So, with DevOps we supply the need of freedom and
more power to execute some tasks which are intrinsically linked to their work"
- P5, Systems Engineer, Spain

A collaboration culture requires a \textbf{product thinking} in substitution to
operations or development thinking. The development team has to understand that
the software is a product that does not end after their push to source code
repository and the operations team has to understand that the process does not
start when an artifact is received for publication. \textbf{Product thinking}
is the third concept related to collaboration culture category.

A gente alterou o perfil profissional das nossas contratacoes. Nos queriamos
contar com pessoas que conseguissem ter uma visao de produto. Nossos arquitetos
conseguiam olhar para o problema e pensar na melhor solucao para ele. Mas ele
tambem tinha o cuidado de olhar para quando aquela aplicacao fosse entrar no ar.
E a gente juntou os desenvolvedores para reforcar que todo mundo tem que pensar
dessa forma. Todo mundo tem que pensar no produto e nao so no seu codigo ou na
sua infraestrutura. P12, Cloud Engineer, United States

There should be a \textbf{facilitated communication} between teams. Ticketing
systems are cited as the typical and inappropriate means of communication
between development and operations. Face-to-face communication is the best
option, but considering that it is not always feasible, the continuous use of
tools like Slack and HipChat was cited as appropriate options.

citar fire

In case of problems at any point in the software life cycle, the responsibility
to check it is shared. The strategy of avoiding liability should be avoided.
The development team can not say that this is a problem in infrastructure, then
it is responsibility of operations team. Or the opposite, the operations team
also can not say that this is a problem in application, then it is
responsibility of development team. A \textbf{blameless} context must exist.
The teams need to be focused on solving problems, not on finding blame and
running away from responsibility. The context of \textbf{shared
responsibilities} involves not only solving problems, but any other
responsibility inherent in the software product must be shared.
\textbf{Blameless} and \textbf{shared responsibilities} are the remaining
concepts of the category.

citar quinto andar

At first glance, consider the creation and strengthening of the collaboration
culture as the most important in DevOps adoption seems somewhat obvious, but
the respondents cited some mistakes that they consider recurrent in not
prioritize this aspect in a DevOps adoption:

"In a DevOps adoption, there is a very strong cultural issue that the teams
sometimes are not adapted. Related to this, one thing that bothers me a lot and
that I see happen a lot is people hitching DevOps exclusively to tools" P9, IT
Manager, Brazil

citar ESX

\subsection{Automation} \label{ssec:automation}
Automation was the category with the higher number of related concepts. This
occurs because manual proceedings are considered as strong candidates to
propitiate the formation of a silo. If a task is manual, one people, or one
team will be responsible to execute it. Although transparency and sharing can
be used to ensure collaboration even in manual tasks, with automation the
points where silos may arise are minimized.

"When a developer needed to build a new application, the common flow was: he
creates a ticket to the operations team, which manually evaluates and creates
what was requested. This task could take a lot of time and there was no
visibility between teams about what was done. So, here, we adopted this
strategy of infrastructure as code where the entire infrastructure is versioned
in a common language that anyone, be it the developer, the operations guy or
even the manager, he looks and says: the configuration of application x is y,
simple". P12, Cloud Engineer, United States

In addition to contribute with transparency, automation is also considered
important to ensure the repeatability of the tasks, reducing rework and risk of
human failure and, consequently, collaborating to increase the confidence
between teams, which is an important aspect of the collaboration culture.

"Before we adopt DevOps, there was a lot of manual work. For example, if you
needed to create a database schema, was manual; if you needed to create a
database server, was manual; if you needed to create one more EC2 instance,
also manually. This manual work was time consuming and often caused errors and
rework" - P1, DevOps Developer, Ireland

"Our main motivation to adopt DevOps was basically reduce rework. Almost every
week, we had to basically make new servers and start it manually, which was
very time-consuming" - P4, Computer Technician, Brazil

We find 8 concepts that were grouped in the automation category:

\begin{itemize}
\item Autonomous services: it is one of the characteristics of microservices,
frequently cited as part of DevOps adoption. The services need to be
independently deployable by fully automated deployment machinery;

\item Containerization: docker is cited as a way to automate the provisioning
the environment where the application will execute: a container. The container
is one frequent way used to automate application execution.

\item Deployment automation: use of tools like Docker, Kubernetes, Arch,
Spinnaker, AWS CodeDeploy, Microsoft Visual Studio Team Service, etc to
automate the deployment task. This automation contributes to repeatability,
parity between environments and to reduce risks of human error in the
deployment task.

\item Infrastructure provisioning automation: use of tools like Terraform,
Ansible, Puppet and Chef to provide infrastructure like database and
application servers in an automated way.

\item Infrastructure management automation: use of tools to management the
infrastructure in an automated way. Examples: management of memory and CPU of
a host, management of versions of libraries and database versioning.

\item Monitoring automation: ability to monitor the application and
infrastructure without human intervention. One classic example is the
automated sending of alerts (SMS, slack message or even cellphone calls) in
case of incidents.

\item Recovery automation: ability to, without human intervention, replace one server instance that presents problem or roolback a failed deployment to the previous version.

\item Test automation: execute the application tests in an automated way.
\end{itemize}

\subsection{Continuous Measurement}
Continuous measurement has two facets in a DevOps adoption process: it can work as an enabler or as an outcome.

As an enabler, perform regularly the activities of measurement and share this responsibility contributes to avoid this typical silo and reinforce the collaboration culture, because it is a typical responsibility of the operations team.

"Citar trechos de entrevista aqui"

As an outcome, continuously collect metrics from applications and infrastructure is a required consequence of DevOps adoption. It occurs because the resultant agility increases the risk of something going wrong. The team should be able to react quickly in case of problems, and the continuous measurement allows it to be proactive and resilient.

"Citar trechos de entrevista aqui"

\begin{itemize}
\item Application log monitoring: every application produces logs, and these logs are an important source of data to be used in the continuous measurement perspective.

\item Continuous infrastructure monitoring: the monitoring is not performed by a specific person or team in a specific moment. The responsibility to monitor the infrastructure is shared and it is executed in day by day software development.

\item Continuous application measurement: the application is instrumented to provide metrics that are used to evaluate aspects and often direct evolutions or business decisions.

\item Monitoring automation: see \ref{ssec:automation}.
\end{itemize}

\subsection{Quality Assurance}
\begin{itemize}
\item Code branching

\item Cohesive services: another characteristic of microservices, frequently cited as part of DevOps adoption. The services need to be focused in doing one thing well, which is one quality assurance aspect.

\item Continuous testing
\item Parity between environments
\item Source code static analysis
\end{itemize}

\subsection{Sharing}
\begin{itemize}
\item Activities sharing
\item Knowledge sharing
\item Process sharing
\end{itemize}

\subsection{Transparency}
\begin{itemize}
\item Infrastructure as code
\item Shared pipelines
\item Sharing on a regular basis
\end{itemize}

\subsection{Agility}
\begin{itemize}
\item Continuous Integration
\item Continuous Delivery
\item Continuous deployment: the deployment is a common point of manifestation of the existence of silos, the development team generates one artifact and throw it \textit{over the wall} to be deployed by the operations team.
\end{itemize}

\subsection{Resilience}
\begin{itemize}
\item Auto scaling
\item Recovery automation
\item Zero down-time
\end{itemize}
