%https://docs.google.com/document/d/1GPGvGlILx7VLJmytuHB63WPelsxnxfXFBNtEj3nlttU/edit?usp=sharing

\documentclass[preprint,12pt,3p]{elsarticle}

\usepackage{booktabs} % For formal tables
\usepackage{xcolor}
\usepackage{xspace}
\usepackage[framemethod=TikZ]{mdframed}
\usepackage{graphicx}
\usepackage{enumerate}

\definecolor{oldlace}{rgb}{0.99, 0.96, 0.9}

\usetikzlibrary{shadows}


\newmdenv [ %
 skipabove=\topsep,
 skipbelow=\topsep,
 leftmargin       = 2              ,
 rightmargin      = 2              ,
 splittopskip     = \topskip      ]{mh}

\newmdenv [ %
 skipabove=\topsep,
 skipbelow=\topsep,
 roundcorner      = 5pt            ,
 leftmargin       = 2              ,
 rightmargin      = 2              ,
 backgroundcolor  = oldlace        ,
 innertopmargin   = 3       ,
 splittopskip     = 3      ]{mq}


\newmdenv [ %
 skipabove=\topsep,
 skipbelow=\topsep,
 shadow=true,
 leftmargin       = 2              ,
 rightmargin      = 2              ,
 splittopskip     = \topskip      ]{mr}

\hyphenation{Dev-Ops}

\newcommand{\cat}[1]{{\textbf{\emph{#1}}}}
\newcommand{\cc}{\cat{collaborative culture}\xspace}

\long\def\gnote#1{[[[[\textbf{#1 -Gustavo}]]]]}


\journal{Journal of Systems and Software}

\begin{document}

\begin{frontmatter}

\title{Adopting DevOps in the Real World: A Theory, a Model, and a Case Study}

\author[label1]{Welder Pinheiro Luz}
\address[label1]{Brazilian Federal Court of Accounts}
\ead{welder.luz@tcu.gov.br}

\author[label2]{Gustavo Pinto\corref{cor1}}
\address[label2]{Federal University of Par\'a}
\cortext[cor1]{I am corresponding author}
\ead{gpinto@ufpa.br}

\author[label3]{Rodrigo Bonif\'acio}
\address[label3]{University of Bras\'{i}lia}
\ead{rbonifacio@cic.unb.br}

\begin{abstract}
DevOps is a set of practices and cultural values
that aim to reduce the barriers between development and operations
teams. Although previous research has tried to
explain DevOps, little is
known about the \emph{practitioners' understanding}
about successful paths for DevOps adoption. In this paper we explore
this issue by presenting a theory, a model, and a DevOps adoption case study.
We use the Classic Grounded Theory approach to characterize
DevOps adoption in 15 companies from different
domains and countries. We propose a model (i.e., a workflow for DevOps
adoption) and evaluate it through
a case study---using a focus group to
better understand an experience with DevOps adoption in a Brazilian
Government institution. This paper increments the existing view of
DevOps by detailing real scenarios and explaining the relevance
of the core components of our theory during DevOps adoption.
We provide evidence that \emph{supporting a collaborative culture} is the main DevOps concern,
contrasting with an existing wisdom that considers that automation and tooling are
sufficient to achieve DevOps. Our results contribute to
build an adequate understanding of DevOps and
to assist other institutions in the path towards DevOps adoption.
\end{abstract}

\begin{keyword}
DevOps \sep Grounded Theory \sep Software Development \sep Software Operations \sep Focus Group
\end{keyword}

\end{frontmatter}

\input{1_introduction}

\section{Research Method} \label{sec:research_method}

We used Grounded Theory (GT) as the research method. GT was
originally proposed by Glaser and Strauss~\cite{glase1967discovery}.
As distinguishing features, it has (1) the absence of clear research hypothesis upfront
and (2) limited exposure to the literature at the beginning of the research. GT
is a theory-development approach (the hypothesis emerge as a result of
a investigation), in contrast with more traditional
theory-testing approaches~\cite{coleman2007using}---e.g., those that
use statistical methods to either confirm or refute pre-established hypothesis.

We used GT as the research method due to three main reasons. First, GT is a consolidated
method in other areas of research - notably medical
sociology \cite{gt_medical_sociology}, nursing \cite{barnsteiner2002using}, education
\cite{gt_education} and management \cite{gt_management}. GT is also being increasingly employed
to study software engineering topics~\cite{hoda2017becoming,stol2016grounded,adolph2011using}. Second,
GT is considered an adequate approach to answer research questions that aims to
characterize scenarios under a personal perspective of those
engaged in a discipline or activity~\cite{barnsteiner2002using},
which is exactly the scenario here: what are the successful adoption paths for DevOps? Finally,
GT allows researchers to build an independent and original understanding,
which is adequate to collect empirical evidence directly from the
practice on industry without bias of previous research. The evidence
is only reintegrated back with the existing literature after the step of
theory construction.

Since the publication of the original version of GT~\cite{glase1967discovery},
several modifications and variations have been proposed to the method, coming to
exist at least seven different versions~\cite{denzin2007grounded}.
Here we chose the classic version, mainly because we did not have a research
question at the beginning of our research, exactly as suggested in this
version. We actually started from an area of interest: successful DevOps adoption
in industry. In addition, research works in software engineering that leverage GT
predominantly use this version~\cite{stol2016grounded}.
We carried out our research using an existing
guideline about how to conduct a
Grounded Theory~\cite{adolph2011using} research. This guideline organizes
a GT investigation in 3 steps: \emph{Open Coding} Data Collection,
\emph{Selective Coding} Data Analysis, and \emph{Theoretical Coding}.

\begin{enumerate}[label=(\Alph*)]
\item {\bf Open Coding Data Collection.} We started our research
  by collecting and analyzing data from companies that claim to have
  successfully adopted DevOps.
  To this end, we have conducted a \emph{raw data analysis} that searches for patterns of
  incidents to indicate concepts,  and then grouped these concepts into
  categories~\cite{stol2016grounded}.

\item {\bf Selective Coding Data Analysis.} In the second step, we evolve
  the initial set of
  categories by comparing new incidents with the previous ones. Selective coding
  starts when a ``core category'' is identified~\cite{stol2016grounded}.
  The core category is responsible for enabling the integration of the other
  categories and structuring the results into a dense and consolidated grounded
  theory~\cite{jantunen2014using}. In selective coding, we only considered the
  specific variables that are directly related to the core category, in order to
  enable the production of an harmonic theory~\cite{coleman2007using,hoda2011impact}.
  Selective coding ends when we achieve a theoretical saturation, which occurs
  when the last few participants provided more evidence and examples but no new
  concepts or categories~\cite{glase1967discovery}.

\item {\bf Theoretical Coding.} After saturation, we built a theory that
explains the categories and the relationships between the categories.
Additionally, we reintegrated our theory with the existing literature, which allowed us to compare our proposal
 with other theories about DevOps. That is, using a Grounded Theory approach,
 one should only conduct a literature review in later stages of a research,
in order to avoid external influences to conceive a theory~\cite{adolph2012reconciling}.

\end{enumerate}

Throughout the process, we wrote memos capturing thoughts and analytic
processes; the memos support the emerging concepts, categories, and their
relationships~\cite{adolph2012reconciling}.

Regarding data collection, we conducted semi-structured interviews with 15 practitioners of companies from
Brazil, Ireland, Portugal, Spain, and United States that
contributed to DevOps adoption processes in their companies. Participants
were recruited by using two approaches: (1) through direct contact in a \emph{DevOpsDays}
event in Brazil and (2) through general
calls for participation posted on DevOps user groups, social networks,
and local communities. In order to achieve a heterogeneous perspective
and increase the wealth of information in the results,
we consulted practitioners from a variety of companies.
Table~\ref{participant_table} presents the characteristics of the participants
that accepted our invitation.
To maintain anonymity, in conformance with the human ethics guidelines,
hereafter we will refer to the participants as P1--P15 (first column). \emph{We
assumed a non-disclosure agreement with the investigated companies to use the
data only in the context of our study and, therefore, we can not disclose them}.

\begin{table}[t]
\centering
\caption{Participant Profile. SX means software development experience in years,
DX means DevOps experience in years, CN means country of work, and CS means
company size (S\textless100; M\textless1000; L\textless5000; XL\textgreater5000).}
\label{participant_table}
\begin{tabular}{p{0.4cm}p{2.6cm}p{0.4cm}p{0.45cm}p{0.5cm}p{1.3cm}p{0.3cm}} \toprule \centering
\textbf{P\#}          & \textbf{Job Title}
       & \textbf{SX} & \textbf{DX} & \textbf{CN}   & \textbf{Domain}    & \multicolumn{1}{l}{\textbf{CS}} \\ \midrule \centering
P1                   & DevOps Developer      & 9            & 2           & IR            & IT                 & S                               \\ \centering

P2                   & DevOps Consult.       & 9            & 3           & BR            & IT                 & M                               \\ \centering

P3                   & DevOps Developer      & 8            & 1           & IR            & IT                 & S                               \\ \centering

P4                   & Computer Tech.        & 10           & 2           & BR            & Health             & S                               \\ \centering

P5                   & Systems Engineer      & 10           & 3           & SP            & Telecom            & XL                              \\ \centering

P6                   & Developer             & 3            & 1           & PO            & IT                 & S                               \\ \centering

P7                   & Support Analyst       & 15           & 2           & BR            & Telecom            & L                               \\ \centering

P8                   & DevOps Engineer       & 20           & 9           & BR            & Marketing              & M                               \\ \centering

P9                   & IT Manager            & 14           & 8           & BR            & IT                 & M                               \\ \centering

P10                  & Network Admin.        & 15           & 3           & BR            & IT                 & S                               \\ \centering

P11                  & DevOps Superv.                & 6            & 4           & BR            & IT                  & M                               \\ \centering

P12                  & Cloud Engineer              & 9            & 3           & US            & IT                  & L                               \\ \centering

P13                  & Technology Mngr.                 & 18            & 6           & BR            & Food                  & M                               \\ \centering

P14                  & IT Manager            & 7            & 2           & BR            & IT                  & S                               \\ \centering

P15                  & Developer        & 3            & 2           & BR            & IT                  & S \\ \bottomrule
\end{tabular}
\end{table}



The interviews were conducted between April 2017 and April 2018 by means of Skype calls
with minimum duration of 20 minutes, maximum of 50 and an average of 31.
Data collection and analysis were iterative so the collected data helped to guide
future interviews. Questions evolved according to
the progress of the research. We started with five open-ended questions: (1) What
motivated the adoption of DevOps? (2) What does DevOps adoption mean in the context of
your company? (3) How was DevOps adopted in your company? (4) What were the
results of adopting DevOps? And (5) what were the main difficulties?

As the analyzes were being carried out, new questions were added to the script.
These new questions were related to the concepts and categories identified in
previous interviews. Examples of new questions include: (1) What is the
relationship between deployment automation and DevOps adoption? (2) Is it
possible to adopt DevOps without automation? (3) How has your company fostered a
collaborative culture?

With respect to \emph{data analysis}, the interviews were
recorded, transcribed, and analyzed. The interviews with participants from
Brazil and Portugal were translated from Portuguese into
English. The first moment of the analysis, called open coding in GT, starts
immediately after the transcription of the first interview.
Open coding lasted until there was no
doubt about the core category of the study. Similar to that described by
Adolph et al.~\cite{adolph2012reconciling}, we started
considering a core category candidate and changed later. The first core category
candidate was \cat{automation}, but we realized that this category did not
explain most of the behaviors or events in data. The sense of
shared responsibilities in solving problems, and the notion of product thinking
are examples of events that could not be naturally explained around \cat{automation}.
We then started to understand that \cc also appeared recurrently in the analysis
and with more potential to explain the remaining events. Thus, we asked the
respondents explicitly about the role of \cat{automation} and how the \cc is
formed in a DevOps adoption process.

Considering the script adaptations and the analysis of new data in a constant
comparison process, taking into account the previous analyses and the
respective memos written during all the process, after the tenth
interview, we concluded that \cc was unequivocally the core
category regarding how DevOps was successfully adopted.
At this moment, the open coded ended and the selective coding started.
We started by restricting the coding only
to specific variables that were directly related to the core category and their
relationships. Following three more interviews and respective analysis, we realized that
the new data added less and less content to the emerging theory. That is, the
explanation around how the \cc category is developed showed signs of saturation.
We then conducted two more interviews to conclude that we had reached a
theoretical saturation, that is, we were convinced there were no more enablers
or outcomes related to DevOps adoption, the relationship between all of them
was adequate and the properties of core category were well developed.

{
\color{blue}
It is important to note that theoretical saturation is closely related to the
questions that one intends to answer in the collection and analysis of the data.
New questions can be asked and new aspects can be studied, demanding new rounds
of iterative data collection and analysis in search of the theoretical
saturation regarding that new aspect.
}

At this point, we started the theoretical coding to find a way to integrate
all the concepts, categories, and memos in the form of a cohesive and
homogeneous theory, where we have pointed out the role of the categories as
enablers and outcomes. We will present more details about
the results of our theoretical coding phase in the next section.
To illustrate the coding procedures, we will show a working example from an
interview transcription to a category.
It is important to note that \emph{raw interview transcripts} are full of noise.
We started the coding by removing this noise and identifying the key points.
Key points are summarized points from sections of the interview~\cite{georgieva2008best}.
For example:

\textbf{Raw data:} \textit{``So, here we have adopted this type of strategy that is
the infrastructure as code, consequently we have the versioning of our entire
infrastructure in a common language, in such a way that any person, a
developer, an architect, the operations guy, or even the manager, he can look
at it and describe that the configuration of application x is y. So, it
aggregates too much value for us exactly with more transparency''}

\textbf{Key point:} \textit{``Infrastructure as code contributes to
transparency because it enables the infrastructure versioning in a common
language to all professionals''}

We then assigned codes to the key point. A code is a phrase that summarizes
the key point and one key point can lead to several codes \cite{hoda2017becoming}.

\textbf{Code:} \textit{Infrastructure as code contributes to transparency}

\textbf{Code:} \textit{Infrastructure as code provides a common language}

In this example, the concept that emerged was ``infrastructure as code''. The
expression corresponding to this concept comes directly from raw data, but this
is not a rule. It is common for the concept to be an abstraction, without
emerging from an expression present in raw data.
At this moment, we already identified other concepts that
contribute to transparency. We wrote the following memo:

\textbf{Memo:} \textit{Similar to sharing on a regular basis and shared
pipelines, the concept of infrastructure as code is an important transparency
related one. These transparency related concepts have often been cited as
means to achieve greater collaboration between teams}.

The constant comparison method was repeated on the concepts to produce a third
level of abstraction called categories. Infrastructure as code was grouped
together with five other concepts into the \textbf{sharing and transparency} category.
Figure~\ref{fig1} illustrates how that abstraction of concepts become a category.


\begin{figure}
  \centering
  \includegraphics[width=0.5\textwidth]{fig1.png}
  \caption{Coding: Building Categories}
  \label{fig1}
\end{figure}


\input{3_categories}

\section{A Theory on DevOps Adoption} \label{sec:results}

The results of a grounded theory study, as the name of the method itself
suggests, are grounded on the collected data, so the hypotheses emerge from
data. A grounded theory should describe the key relationships between the
categories that compose it, i.e., a set of inter-related hypotheses~\cite{hoda2017becoming}.
We present the categories of our grounded theory
about DevOps adoption as a network of the two categories of enablers (\cat{automation},
\cat{sharing and transparency}) that are commonly used to develop the core category
\cc, as discussed in the previous section. Based on our understanding,
implementing the enablers to develop the \cc typically leads
to concepts related to two categories of expected outcomes:
\cat{agility} and \cat{resilience}. Moreover, there are two categories that can be considered
both as enablers and as outcomes: \cat{continuous measurement} and \cat{quality assurance}.
In this section, we describe the relationships between those categories, building a theory
of DevOps adoption.

\subsection{A General Path for DevOps Adoption}

In Section~\ref{sec:method} we presented the general questions of this
research, including: How do practitioners characterize a successfully path
   for DevOps adoption? Here, we elaborated a response to this question,
based on our grunded theory study. The main
point that should be formulated is the construction of a \cat{collaborative
culture} between the software development and operations teams and
related activities. According to our findings, the other categories,
many of which are also present in other studies that have investigated DevOps,
only make sense if the practices and concepts related to them either contribute
to the level of a \cc or lead to the expected consequences
of a \cc. This leads to a general hypothesis of our work, which is:

\begin{mh}
  {\bf General Hypothesis:} \textit{DevOps brings several benefits
  related to Agility, Resilience, and Software Quality
  Assurance. Nonetheless, to achieve this benefits,
  it is highly recommended to work towards a} \cc,
  \textit{removing silos and and implementing a more direct
  communication between the teams}. 
\end{mh}

The quote bellow makes explicit the relevance of a \cc
---as well as the challenges to keep this culture-to
achieve the benefits with DevOps adoption.

\begin{mq}
  ``\emph{Keeping (the collaborative) culture alive is still a challenge for us,
  and we consider it very important. Here in the company, for example, we have
  tech talks that are monthly conversations with the teams. The
  purpose of these Tech Talks is to share knowledge, technologies, and work
  procedures, by increasing the transparency about how everything works. We also
  have a DevOps culture Slack channel, where we discuss DevOps. The idea
  is not to let the culture die, \ldots, because that is the essence of
  DevOps for us}.'' (P12, Cloud Engineer, United States)
\end{mq}


This hypothesis emerged from our understanding about the
perceptions of the practitioners on the concepts that might
positively influence the adoption of DevOps. We built
this understanding iteratively, through discussions
considering the transcripts, memos, categories and their relationships. Surelly, during this
process, different opinions emerged, though in the end
the authors of this paper agreed with this general hypothesis.
For intance, in certain moments of the research, we used to believe that DevOps would
actually mean the \emph{end of the operations teams}. After
several rounds of discussion, we concluded that DevOps actually
focus on a collaborative approach for executing development and
operation tasks. This understanding induces four sub-hypothesis, as discussed in what
follows.

\begin{mh}
\textbf{Hypothesis 1:} \textit{Certain categories related to DevOps adoption
only make sense if used to increase the} \cc \emph{level. We
call this set of categories of \textbf{enablers}}.
\end{mh}

Based on this hypothesis, the maturity of DevOps adoption does not
advance in situations where only one team is responsible to understand, adapt, or
evolve automation---even when such automation supports different activities like
deployment, infrastructure provisioning and monitoring. The same holds for the
other \emph{enabling} categories. That is, in situations that
\cat{transparency and sharing} do not contribute to
the \cc, they do not contribute to DevOps adoption as a whole. This is
clear when one of the participants of our study states that

\begin{mq}
``\emph{DevOps involves tooling, but DevOps is not tooling. That is, people often
focus on using tools that are called `DevOps tools', believing that this is
what DevOps is. I always insist that DevOps is not tooling, DevOps involves the
proper user of tools to improve software development procedures.}'' (P2, DevOps
Consultant, Brazil)
\end{mq}

\begin{mh}
\textbf{Hypothesis 2:} \textit{Some other categories are not related to DevOps
adoption for contributing to increase the} \cc \emph{level, but insted for emerging
as an expected or necessary consequence of the adoption. These 
represent the set of \textbf{outcome} categories}.
\end{mh}

In a first moment, the simple fact that a team is more
\cat{agile} in delivering software, or more \cat{resilient} in failure recovery, does not
contribute directly to bringing operations teams closer to development teams.
Nevertheless, a signal of a mature DevOps adoption is an increasing capacity for continuously
delivering software (and thus being more \cat{agile})
and for building \cat{resilient} infrastructures. 

\begin{mh}
\textbf{Hypothesis 3:} \textit{The categories \cat{Continuous Measurement} and \cat{Quality Assurance}
  are both related to DevOps enabling capacity and to expected
  DevOps outcomes}.
\end{mh}

Measurement is cited as a typical responsibility of the operations team.
At the same time that sharing this responsibility reduces silos,
it is also cited that measurement is a necessary consequence of DevOps adoption. Particularly because
the continuous delivery of software requires more control,
which is supplied by concepts related to the \cat{continuous measurement} category.
The same premise is valid to the \cat{quality assurance} category. At first glance,
\cat{quality assurance} appears as one response to the context of agility in operations
as a result of DevOps adoption. But, the efforts in quality assurance of software products
increase the confidence between the development and operations teams, increasing the level
of \cc.

Altogether, DevOps enablers are the means commonly used to increase the level of
the \cc in a DevOps adoption process.
We have identified five categories of DevOps enablers:
\cat{ Automation}, \cat{Continuous Measurement}, \cat{Quality Assurance},
\cat{Sharing}, and {\cat{Transparency}. Another finding of our
study leads to our fourth hypothesis.

\begin{mh}
\textbf{Hypothesis 4:} \textit{There is no precedence between enablers in a DevOps adoption process}.
\end{mh}

We have realized that the adoption process might not have
to prioritize any enabler, and a company that aims to implement
DevOps should start with  the enablers that seem more appropriate (in terms
of its specificities). Accordingly, we did not find any evidence that an enabler
is more efficient than another for creating a \cc. \cat{Automation} is the category
that appears more frequently in our study, though several participants make
clear that associating DevOps with automation is a misconception.

%\begin{mq}
%``\emph{I think that the expansion of collaboration between teams involved other
%things. It was not just automation. There must be an alignment with the
%business needs. (...) I think that DevOps enabled a broader understanding
%of software production and we realized the very fact that it is not about
%automating everything. (...) So, I see with caution a supposed vision that
%automating things can be the way to implement DevOps.}'' (P7, Support Analyst, Brazil)
%\end{mq}

DevOps outcomes are the categories that does not primarily produce the
expected effect of an {\bf enabler}, typically concepts that are expected as
consequences of an adoption of DevOps. We have identified four categories that
can work as DevOps outcomes: \cat{agility}, \cat{continuous measurement},
\cat{quality assurance}, and \cat{software resilience}. Note that,
as mentioned before, \cat{continuous measurement} and \cat{quality assurance}
are both enablers and outcomes.

That is, a well succeeded DevOps adoption typically increases the potential of
\cat{agility} of teams and enables \cat{continuous measurement}, \cat{quality assurance} and
\cat{resilience} of applications.
However, in some situations, this potential is not completely used due to business
decisions. For example, one respondent cited that, at a first moment, the
company did not allow the continuous deployment (more potential of agility)
of applications in production.

%\begin{mq}
%``\emph{We had conditions and security to continuously publish in production,
%however, in the beginning, the managers were afraid and decided that the
%publication would happen weekly.}'' (P9, IT Manager, Brazil)
%\end{mq}

Considering the hypothesis we build from our understanding about
a successfully path for DevOps adoption, we answer our
first research question ``\emph{(RQ1) How do practitioners characterize a successfully path
for DevOps adoption?}''

\begin{mr}
  \textbf{(RQ1) Answer:} In order to successfully
  conduct an effort for DevOps adoption, practitioners
  suggest the focus on building a \cc (this is
  the essence of DevOps), which should be achieved through Automation,
  Transparency and Knowledge Sharing, Continuous Measurement,
  and Quality Assurance. Without setting up \cc as the main
  goal, the chances of failing to achieve the expected
  benefits of adopting DevOps (e.g., Agility and Resilience)
  increase. 
\end{mr}



\subsection{Outline of the Theory on DevOps Adoption}

We summarize our theory in this section using a 
set of recommendations about building and reporting
theories in software engineering~\cite{sjoberg2008}.
Accordingly, a theory should be described in
terms of {\bf constructs},
{\bf propositions}, propositions' {\bf explanation},
and theory {\bf scope}.

The main construct of our theory is \emph{DevOps adoption},
which means any effort for building a \cc between
the development and operations teams. DevOps adoption is
supported by other constructs, including \emph{automation}
(deployment automation, infrastructure provision automation,
test automation, and so on) and knowledge sharing (e.g.,
sharing procedures and making clear the task assignments
of the teams). We expanded our hypothesis to identify
a set of nine propositions, which we can explain
by means of the categories, categories' relations,
and transcription memos. For instance, the necessity of
using tech talks, simplifying communication procedures, and employ
tools like Slack and Hip Chat explains the relevance of knowledge sharing
(P2). Similarly, we can explain proposition (P7) by 
considering the transcription memo ``\emph{DevOps reduces (or even eliminates) downtime}''. 


\begin{enumerate}[(P1)]
 \item The use of automation enables DevOps
 \item The use of practices and tools for sharing knowledge enables DevOps
 \item The use of continuous measurement procedures enables DevOps
 \item The use of quality assurance methos enables DevOps
 \item DevOps centered on \cc increases the agility of the teams
 \item DevOps centered on \cc increases systems' resilience 
 \item DevOps centered on \cc decrease systems'  downtime
 \item DevOps centered on \cc supports continuous measurement 
 \item DevOps centered on \cc supports SQA activities   
\end{enumerate}

We represent the elements of our theory using the diagram
of Figure~\ref{}. 





\input{5_model}

\section{Application of The Model} \label{sec:tcu}

{
\color{blue}
In this section we detail an real experience on DevOps adoption where we could
demonstrate the practical application of the model proposed above.
}

\subsection{Context}

The TCU is responsible for the accounting, financial, budget, performance, and property
oversight of federal institutions and entities of the country. Currently, there are 2500
professionals working at the TCU, of which approximately 300 work directly on either
software development or operations. The source code repository at the TCU hosts,
more than 200 software project, totaling over 4 million lines of code.

{
\color{blue}
For many years, the TCU software development process was marked by a strong call
for standardization of procedures. This approach favored the specialization of
activities, causing segregation of development and operations teams in a rigid
silo structure. Work on this structure revolved around bureaucratic
communication and well-defined service level agreements - SLAs. Communication
between teams occurred basically through the use of a corporate service-desk tool.
The SLAs provided maximum deadlines for the operations team to perform the
procedures requested by the development team through the service-desk. The
procedures include provisioning of infrastructure and database servers for
development, testing, staging and production environments, requesting access
to server logs, and the most controversial of all: deployment of new versions
of applications. The deployment involved a long chain of a week-long process,
whose complex chaining of responsibilities often ran out of schedule, causing
months of delays in software deliveries.

In addition to delays in software deliveries, the silo structure caused a
devastating blaming game between the teams. If there were delays in delivery
or errors in some functionality, the focus was not on solving the problem, but
on pointing out which team was the culprit.

In search of solutions to these problems, and knowing the alleged benefits of
the approach, the TCU included the development of the DevOps approach among the
objective of its IT area. The first attempt by the technical teams to reach the
goal was to seek consulting firms whose proposed solutions invariably involved
implanting specific DevOps tools. These DevOps tools were typically related to
automate some aspect of the process and, naturally, the teams turned their
attentions to establish clear and separated responsibilities to each aspect of
the tools. Obviously, these DevOps tools only changed the points of conflict.
In this context, our research was developed and, after that, we applied our
model aiming to increase the DevOps level at TCU. In this section we described
this experience.
}

\subsection{Applying The Model}

Before the application of our model, the TCU had produced some w.r.t deployment
automation results and the focus was being directed to the tooling issue. Considering this
incomplete perspective of DevOps, the conflicts between development and operations
teams continued. That is, the mere advance in implanting ``DevOps tools'' simply
changed the points of conflict, but they persisted.

After the presentation of our  model in a series of lectures, development and
operations teams changed their focus to build a \cc. This
change was only possible due to the engagement and sponsorship of the IT
managers. Looking to the concepts within the \cc category, the first practical
action at the TCU was to facilitate communication between teams. The use of tickets
was then abolished. The problems had to be solved in a collaborative way, preferably
face to face.
Looking to enablers, the TCU is applying \cat{sharing and transparency} concepts.
The role of internal tech talks and committees to disseminate that collaboration
culture and related concepts is being reinforced.
When a new infrastructure had to be provided and configured, the current guideline is
that there must be a kind of \emph{pair programming} between developers and infrastructure
members. All application related tasks must be executed in a collaborative
way. Naturally, the professionals noticed that automation would facilitate the
operationalization of that collaboration. For this reason, the infrastructure provisioning
and management was automated.

The TCU also uses continuous measurement and quality assurance concepts as
enablers of its DevOps adoption. The applications started to be continuously
tested and measured. The tests were automated and included in the pipelines.
Verification of test coverage and quality code also became part of the pipeline.
This increased the confidence between teams. The TCU started
to explore the potential of DevOps tools, like recovery automation, zero
down-time, and auto scaling. The deployment has also been automated.
It is important to note that, before DevOps, deployment activities were historically a controversial point at the TCU.
Several conflicts occurred over time. Rigid procedures were created to try to
avoid problems. These ``rigid procedures'' often led to periods of months
without any software delivery. The more collaborative scenario, with a strong appeal in automation and quality,
created by following an appropriate path in adopting DevOps, enabled the deployment activities to become
a lightweight task at the TCU. Continuous deployment became a reality and, currently, several deployments
occur as regular activities of the development teams at the TCU.

Since the TCU is a government institution, some advances in DevOps adoption still comes up
against regulatory issues. For example, there are internal regulations that
establish that only the operations sector is responsible for issues related to
application infrastructure, contrasting with shared responsibilities that are
part of the \cc. Nevertheless, our model enabled the TCU to adopt DevOps in a more
sustainable way. Knowing the role
of each DevOps element in the adoption was fundamental for the TCU to avoid points
of failure and to build a collaborative environment that supports the
exploration of DevOps benefits.

\subsection{A Focus Group}


\section{Threats to Validity}\label{sec:threats}

Regarding construct validity,  we are actually relying on the subjective
practitioners' perception when we stated that we performed our study considering successful cases
of DevOps adoptions. However, currently, there is no objective way to measure whether or not a
DevOps adoption was successful.
Although Grounded Theory offers rigorous procedures for data analysis, our
qualitative research may contain some degree of research bias. Certainly, other
researchers might form a different interpretation and theory after analyzing
the same data, but we believe that the main perceptions would be preserved.
This is a typical threat related to GT studies, which do not claim to generate
definitive findings. The resulting theory, for instance, might
be different in other contexts \cite{hoda2012developing}.

For this reason, we do not claim
that our theory is absolute or final. We welcome extensions to the theory based
on unseen aspects or finer details of the present categories or potential discovery
of new dimensions from future studies.
Future work can also focus on investigating contexts
where DevOps adoption did not succeed, aiming to validate if our model could be
relevant in this scenario too. Finally, regarding external validity, although we
considered in our study the point of view of practitioners with different
backgrounds, working in companies from different domains, and distributed across
five countries, we do not claim that our results are valid for
other scenarios---although we almost achieved saturation
after the 12$^{th}$ interview. Accordingly, our degree of heterogeneity complement
previous studies that mostly focus in a single company (as we will discuss next).

The focus group was moderated by one of the researchers, the participants were
arbitrarily invited, without a general call, and they are co-workers of one of
the researchers. Although they were chosen arbitrarily, the choice was made
precisely by the prior knowledge of which professionals were directly involved
in DevOps adoption at TCU. To mitigate this threat, the participants were
informed the purpose of the group was to obtain an evaluation of the DevOps
adoption as a whole and that they had total freedom to expose their real
opinions, whether they were favorable or not to the implanted model.


\section{Related Work} \label{sec:related_work}

The research literature is particularly rich when it comes to DevOps-related
works (e.g.,~\cite{devops_a_definition_xp_15,extending_dimensions_icsea_16,qualitative_devops_journalsw_17}).
In a literature review, Erich et al.~\cite{cooperation_dev_ops_esem_2014} presents 8
main concepts related to DevOps: culture, automation, measurement, sharing,
services, quality assurance, structures and standards. The authors pointed out
that the first four concepts are
related to the CAMS framework, proposed by Willis~\cite{what_devops_means_2010}.
The paper concludes that there is a great opportunity for empirical researchers
to study organizations experimenting with DevOps.
Other studies (e.g.,~\cite{devops_a_definition_xp_15,dimensions_of_devops_xp_15,extending_dimensions_icsea_16,characterizing_devops_sbes_2016,qualitative_devops_journalsw_17})
mixed literature reviews with empirical data to investigate DevOps.
Although our research and recent literate are interested in understanding DevOps,
there are subtle differences in both (1) the methodological aspects and (2) the focus
of each work.

First of all, none of the aforementioned works focused on explaining the process of DevOps adoption,
in particular, using data collected in the industry. This is unfortunate, since the
practitioners' perception present an unique point of view that researchers
alone could hardly grasp. Moreover, although the literature has a number of
useful elements, there is a need to complement such elements with a perspective on how DevOps has
been adopted, containing guidance about how to connect all these isolated parts
and then enabling new candidates to adopt DevOps in a more consistent way.
For instance, the work of Erich et al.~\cite{qualitative_devops_journalsw_17}
focus on investigating the ways in which organizations implement DevOps.
However, this work relies only in literature review and does not formulate
new hypothesis about DevOps adoption. Second,
in terms of results, our main distinct contribution is to improve the guidance
to new practitioners in DevOps adoption.
Next, we present the overlappings of our
results with the existing literature, presenting also the main differences that
make the contributions of our work clearer.

The work of Smeds et al.~\cite{devops_a_definition_xp_15} uses a literature
review to produce one explanation about DevOps through a set of enablers and capabilities. Additionally, their results
present a set of impediments of DevOps adoption based on an interview with 13
subjects of a same company, and whose DevOps adoption process was at
an initial stage. The main similarities with our study are: (1) grouping
elements as DevOps enablers; and (2) the presence of several similar concepts:
(a) testing, deploying, monitoring, recovering and infrastructure automation;
(b) continuous integration, testing and deployment; (c) service failure recovery
without delay; and (d) constant, effortless communication. The main
differences are: (1) their work does not group concepts into categories,
for example: most of their enablers were grouped together by us within the \cat{automation} category; (2) presents cultural enablers as
common contributor to DevOps, not as the most important concern; and (3) the empirical
part of the study focus on building a list of possible impediments to DevOps
adoption, not on providing guidance to new adopters.


In the study of Lwakatare et al.~\cite{dimensions_of_devops_xp_15}, the
authors aimed at characterizing the and formalize what DevOps is about. Through
a sequence of interviews, the authors observed the need of four dimensions to compose
DevOps, including collaboration, automation, measurement, and monitoring.
In a follow up study,
Lwakatare et al.~\cite{extending_dimensions_icsea_16} proposed a conceptual
framework to explain ``DevOps as a phenomenon''. The framework is organized around
five dimensions (collaboration, automation, culture, monitoring and measurement)
and these dimensions are presented with related practices.
These two works have good similarities with our study. For instance,
all aforementioned dimensions are also presented here. The
main differences are: (1) collaboration and culture are presented by us
as a single abstraction; (2) Concepts related to monitoring and measurement are
grouped by us in a single category: \cat{continuous measurement}; and (3) it does
not indicate a major dimension (aka, the core category).
Moreover, our work greatly expand the notion of DevOps, proposing a theory
for adoption, and indeed applying this theory in a real setting.

Fran\c{c}a et al.~\cite{characterizing_devops_sbes_2016} present a DevOps
explanation produced by means of a multivocal literature review. The data was collected
from multiple sources, including gray literature, and analyzed by using procedures
from GT. The results contain a set of DevOps principles, where
there is most of the overlapping with our study. In addition, the paper
presents a definition to DevOps, issues motivating its adoption, required skills,
potential benefits and challenges of adopting DevOps. The main similarities
are: (1) Automation, sharing, measurement and quality assurance are presented as
DevOps categories; and (2) Their social aspects category is similar to our
\cc category. The main differences are: (1) it presents DevOps as a
set of principles, different from enablers and outcomes in our study; and (2) the Leanness
category is not present in our study and the \cat{resilience} category is not present
in theirs; and (3) it does not indicate a core category.

The study conducted by Erich et al.~\cite{qualitative_devops_journalsw_17},
similarly to the others cited above, combined literature review with some
interviews with practitioners. In the literature review part, the papers were
labeled and the similar labels are grouped. The 7 top labels are then presented
as elements of DevOps usage in literature: culture of collaboration, automation,
measurement, sharing, services, quality assurance and governance. After the literature
review, six interviews were conducted in order to obtain evidence of DevOps
adoption in practice. The interviews were analyzed individually and a comparison
between them was made, focusing on problems that organizations try to solve by
implementing DevOps, problems encountered when implementing DevOps and practices
that are considered part of DevOps. The main similarity with our study
is that 5 of their 7 groups are also present in our study (culture of collaboration,
automation, measurement, sharing and quality assurance). The main
differences are: (1) it does not consolidate the practitioners' perspective, but
only compare it with literature review results; and (3) it does not indicate a major group.

Finally, the work of Vergori and colleages~\cite{Vergori:2017:ICPE}, the
authors proposed a set of metrics related to DevOps performance, such as
expected task completion rate, expected finishing time, and the proportion of
time doing dev or ops activities. Although some of the metrics proposed are
reasonable straightforward to measure (e.g., the task completion rate), some
other are not so easy. In this work, the authors used the Phoenix project~\cite{Kim:2013:DevOps}, which
is a case of industrial DevOps adoption. In terms of similarities to our work,
both works focus on improving DevOps experience. While we focus on DevOps
adoption, their work focus on DevOps performance.

In comparison with our previous paper~\cite{Luz:2018:ESEM}, here we advance
further in exploring real scenarios of DevOps usage. The TCU scenario was
described in details through the results of the focus group. Besides that, we
present a more detailed explanation about the application of our model,
highlighting each step and presenting some numbers.


\section{Final Remarks} \label{sec:conclusion}

In this paper, grounded in data collected from successfully DevOps adoption
experiences, we present a theory on DevOps adoption, a model of how to adopt
DevOps according to this theory, and a case of applying it in practice.

We found out that the DevOps adoption involves a very specific relationship between
seven categories: \cat{agility}, \cat{automation}, \cc, \cat{continuous
measurement}, \cat{quality assurance}, \cat{resilience}, \cat{sharing and transparency}.
The core category of DevOps adoption is the \cc. Some of the
identified categories (i.e., automation and sharing and transparency) propitiate
the foundation of a \cc. Other categories
(i.e., agility and resilience) are expected consequences of this formation.
Finally, two other categories (i.e., continuous measurement and quality
assurance) work as both foundations and consequences. We call the foundations
categories ``DevOps enablers'', and the consequences categories ``DevOps outcomes''.
Crucially, this model simplifies the understanding of the
complex set of elements that are part of DevOps adoption, enabling it to be
more direct and to offer a lower risk of focusing on wrong things.

We experimented with
this model in real settings, improving the benefits of adopting DevOps
within a government institution that faced many problems with the separation between the
development and operations teams.



\bibliographystyle{elsarticle-num}
\bibliography{references}

\end{document}

%%
%% End of file `elsarticle-template-num.tex'.
