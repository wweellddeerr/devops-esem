\section{Introduction} \label{sec:introduction}
% no \IEEEPARstart

DevOps is a a set of practices and cultural values that has emerged in the
software development industry. Even before
the existence of the term --- a mix of ``development" and ``operations"
words~\cite{httermann2012devops} --- companies like Flickr~\cite{flickr}
had already pointed out the need to break the existing separation between
the operations and software development team. Since then, the term
has appeared without clear delimitation and gained strength and interest
in companies that perceived the benefits of applying agile practices in
operation. DevOps related benefits include increased organizational IT
performance and productivity, cost reduction in software lifecycle, improvement
in operational efficacy and efficiency, and greater business alignment between
development and operations teams~\cite{characterizing_devops_sbes_2016}.
%tasks\gnote{since you are
%claiming a benefit, it is worth to point out a reference.}.
However, the adoption of DevOps is still a challenging task, because there is a
plethora of information, practices, and tools related to DevOps, but it is still unclear
how could one leverage such rich yet scattered information in a organized and
structured way in order to properly adopt DevOps.
%\gnote{why it is challenging?}.
% Welder: tentei apresentar uma justificativa razoável.

% Welder: inseri esse parágrafo para tentar melhorar a justificativa para
% a pesquisa. Não sei se essas citações de reports são adequadas.

Some industrial reports have reinforced the importance of DevOps and its
growing interest. For instance, a recent survey with 1,492 practitioners from
all over the world revealed
that 71\% of the respondents are either implementing DevOps or
planning to adopt DevOps in the subsequent 12 months~\cite{state_of_agile}.
Another report claimed that DevOps enables mission achievement
for any type of organization, independent of industry or sector~\cite{state_of_devops}.
Still, another study reported that DevOps is one of the trends in
today's software industry~\cite{stackoverflow_2018}.

In the academic arena, based on literature reviews, researchers have proposed a
number of DevOps characterizations, for instance, as a set of concepts with
related
practices~\cite{cooperation_dev_ops_esem_2014,devops_a_definition_xp_15,dimensions_of_devops_xp_15,extending_dimensions_icsea_16,characterizing_devops_sbes_2016,qualitative_devops_journalsw_17}. Although some
of these studies leverage qualitative approaches to gather practitioners perception (for instance,
conducting interviews with them), they focus on characterizing DevOps,
instead of providing some guidance to assist on DevOps adoption. Consequently,
the obtained DevOps characterizations allow a comprehensive understanding of
the elements that constitute DevOps, but do not provide detailed guidance to
support newcomers interested in adopting DevOps practices.
As a consequence, many practical and timely questions still remain open, for
instance: (1) Is there any recommended path to adopt DevOps? (2) Since
DevOps is composed by multiple elements~\cite{dimensions_of_devops_xp_15}, do
these elements have the same relevance, when adopting DevOps?
%For example, according to the framework proposed by Lwakatare and colleagues~\cite{dimensions_of_devops_xp_15}, automation is one of four dimensions.
%So, if, by example, one team automated all of your procedures, will be him reached 25\% of DevOps adoption?
(3) What is the role played by elements such as measurement, sharing, and automation
in a DevOps adoption? To provide answers to these questions, we need a holistic
understanding of the practices that successful DevOps teams adopted.

In this paper, we present a model based on the perceptions of practitioners from
15 companies across five countries that successfully adopted DevOps. The model
was constructed based on a classic Grounded Theory (GT) approach. After its
construction, we instantiated the model in the Brazilian Federal Court of
Accounts (hereafter TCU), a Brazilian Federal Government institution. TCU was
bogged down in implanting specific DevOps tools, repeating the same non-DevOps
problems, with conflicts between development and operations teams about how to
divide the responsibilities related to the tools.
% rather than the goalas a mean and not an end and, consequently, better explore the benefits of
%these tools in a more collaborative environment.
%\gnote{explicar brevemente como foi feito}

%Considering that we start from a limited exposure to existing literature, due
%to the nature of a GT approach, this study validates the previous research in
%the parts where they coincide

%\gnote{nao entendi bem a sentença. começa falando de literatura limitada, mas
%depois conclui dizendo validaria a literatura. ou seja, seja a literatura
%limitada ou não, acabaria sendo validada da mesma forma, certo?}.

%Welder: Tem toda razão Gustavo, o que eu disse não faz sentido.

%We found elements that make up the adoption of DevOps and
%most of them are similar to those present in related work.

We found that the DevOps adoption involves a very specific relationship between
eight key categories: agility, automation, collaboration culture, continuous
measurement, quality assurance, resilience, sharing, and transparency. The core
category of DevOps adoption is the collaboration culture. Some of the
identified categories (i.e., automation, sharing, and transparency) only exist
to propitiate the foundation of a collaboration culture. Other categories
(i.e., agility and resilience) are expected consequences of this formation.
Finally, two other categories (i.e., continuous measurement and quality
assurance) work as both foundations and consequences. We call the foundations
categories as ``DevOps enablers'', and the consequences categories as ``DevOps outcomes''.
This way of guidance simplify the understanding of the
complex set of elements that are part of DevOps adoption, enabling it to be
more direct and with lower risk of failure\gnote{temos evidencia que isso poderia diminuir o risco de falha?}.


The model made clear that practitioners interested in adoption DevOps should focus on building a
collaboration culture, which prevents common pitfalls related to focusing on
tooling or automation. When instantiated,
our model helped TCU to
change its focus to improve the collaboration between teams, and to use the tooling
to support (rather than being the goal of) the process.
%In addition, the model explains the common ways used by
%practitioners to reach this development of a collaboration culture: we call this
%understanding as ``DevOps enablers''. Finally, the model explains the role of some concepts related to
%DevOps adoption but that do not contribute directly to bring a collaboration culture:
%the DevOps outcomes.


%\gnote{acho que ta
%faltando um take away mais objetivo aqui. parte da argumentação da existencia
%desse trabalho é que ele pode facilitar a adoção de devops por praticantes da
%industria. de que forma isso seria possivel? seria bom detalhar aqui}
%
% Welder: tentei atender à recomendação com o texto do parágrafo anterior.

The main contributions of this paper are the following:

\begin{itemize}
\item A model, based on the classic Grounded Theory approach, that could support practitioners interested in adopting DevOps based on evidence acquired from their industry peers;
\item An instantiation of this model in a real world, non-trivial context: a Brazilian Federal Government institution.
%\item The use of the classic Grounded Theory approach in a DevOps context, which had not been done before.
\end{itemize}


%The rest of the paper is structured as follows: section \ref{sec:related_work} summarizes the related work. In section \ref{sec:research_method}, the application of Grounded Theory in this study is described with examples. Section \ref{sec:results} contains the results with detailed description of how to adopt DevOps. Section \ref{sec:categories_concepts} presents details about categories and related concepts; In section \ref{sec:case_study} is presented a case study that demonstrates the potential application of the results. The paper concludes in section \ref{sec:conclusion}.
