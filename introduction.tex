\section{Introduction}
% no \IEEEPARstart

DevOps is a term that has emerged in software development industry. Even before
the existence of the term - a mix of two words, development and operations that
has been used since 2009 - companies like Flickr \cite{flickr} already
pointed out the need to break the existent paradigms about the role of
operations in software development teams. Since then, the term that has appeared
without clear delimitation has gained strength and interest in companies that
can perceive the benefits of apply agile practices in operation tasks too.
However, adopt DevOps still as a challenge task.

Researchers have proposed a number of characterizations of DevOps as a set of
concepts with related practices, and using predominantly literature reviews
\cite{cooperation_between_esem_14}, \cite{devops_a_definition_xp_15},
\cite{dimensions_of_devops_xp_15}, \cite{extending_dimensions_icsea_16},
\cite{characterizing_devops_sbes_2016}, \cite{a_qualitative_study_journal_sw_17}
and **OUTROS***. Despite of some these studies include some interviews with industry
practitioners, the interviews are used merely to complement the literature
review data, not having a leading role. Consequently, the obtained DevOps
characterizations allow a comprehensive understanding of the elements that
constitute it, but they don't provide detailed guidance to new candidates that
want to adopt DevOps. This implies several open questions: Is there a suitable
way to adopt DevOps? Do the different elements of DevOps have the same weight
in a DevOps Adoption? For example, according to the framework proposed in
\cite{dimensions_xp_2015} automation is one of four dimensions. So, if, by example,
one team automated all of your procedures, will be him reached 25% of DevOps
adoption? Where, why and how elements like measurement, sharing, automation and
others appear in a DevOps adoption? To answer these questions, it is necessary a
holistic understanding of how software development teams that have succeeded
adopted DevOps did this.

In this paper, we present a model of how DevOps was succeeded adopted in 15
practitioners from companies across five countries. The model was constructed
based on a Grounded Theory study and was applied in an important Brazilian
federal government institution. We find the elements that make up the adoption
of DevOps and most of them are similar to those present in related work.
Considering that we start from a limited exposure to existing literature,
due the nature of a grounded theory study, this study validates the previous
research in the parts where they coincide.

We find that DevOps adoption involves a very specific relationship between
eight elements that have assumed the form of Grounded Theory categories on
this study: agility, automation, code quality and organization, collaboration
culture, continuous measurement, resilience, sharing and transparency. The
core category and validation point of DevOps adoption is collaboration culture.
One part of the identified elements (automation, continuous monitoring, sharing
and transparency) exists as means typically used to propitiate the formation
of a collaboration culture and the another part of elements are consequences
of this formation (code quality and organization, agility and resilience).

The rest of the paper is structured as follows: section II summarizes the
related work. In section III, the application of Grounded Theory in this study
is described with examples. Section IV contains the results with detailed
description of adoption elements and relationships. In section V is presented
a case study that demonstrates the potential application of the results. The
paper concludes in section VI.
