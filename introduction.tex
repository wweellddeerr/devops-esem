\section{Introduction}
% no \IEEEPARstart

DevOps is a term that has emerged in software development industry. Even before
the existence of the term - a mix of "development" and "operations" words, that
has been used since 2009 - companies like Flickr \cite{flickr} already
pointed out the need to break the existent paradigms about the role of
operations in software development teams. Since then, the term that has appeared
without clear delimitation has gained strength and interest in companies that
can perceive the benefits of apply agile practices in operation tasks too.
However, adopt DevOps still as a challenge task.

Researchers have proposed a number of characterizations of DevOps as a set of concepts with related practices, and using predominantly literature reviews \cite{cooperation_dev_ops_esem_2014}, \cite{devops_a_definition_xp_15}, \cite{dimensions_of_devops_xp_15}, \cite{extending_dimensions_icsea_16}, \cite{characterizing_devops_sbes_2016} and \cite{qualitative_devops_journalsw_17}. Despite of some these studies include some interviews with industry practitioners, the interviews are used merely to complement the literature review data, not having a leading role. Consequently, the obtained DevOps characterizations allow a comprehensive understanding of the elements that constitute it, but they don't provide detailed guidance to new candidates that want to adopt DevOps. This implies several open questions: Is there a suitable way to adopt DevOps? Do the different elements of DevOps have the same weight in a DevOps Adoption? For example, according to the framework proposed in \cite{dimensions_of_devops_xp_15} automation is one of four dimensions. So, if, by example, one team automated all of your procedures, will be him reached 25\% of DevOps adoption? Where, why and how elements like measurement, sharing, automation and others appear in a DevOps adoption? To answer these questions, it is necessary a holistic understanding of how software development teams that have successfully adopted DevOps did this.

In this paper, we present a model of how DevOps was successfully adopted in 15 companies across five countries. The model was constructed based on a Grounded Theory study and was applied in an important Brazilian federal government institution. We find the elements that make up the adoption of DevOps and most of them are similar to those present in related work. Considering that we start from a limited exposure to existing literature, due the nature of a grounded theory study, this study validates the previous research in the parts where they coincide.

We find that DevOps adoption involves a very specific relationship between eight categories: agility, automation, collaboration culture, continuous measurement, quality assurance, resilience, sharing and transparency. The core category and validation point of DevOps adoption is collaboration culture. One part of the identified categories (automation, sharing and transparency) exists as means typically used to propitiate the formation of a collaboration culture. Another part of categories are expected or required consequences of this formation (agility and resilience). And, other two categories (continuous measurement and quality assurance) can work in the two ways mentioned above.

The main contributions of this paper are: improve the guidance about how do adopt DevOps based on an industry view; provide one detailed example of Grounded Theory use in a DevOps investigation, which had not been done before.

The rest of the paper is structured as follows: section \ref{sec:related_work} summarizes the related work. In section \ref{sec:research_method}, the application of Grounded Theory in this study is described with examples. Section \ref{sec:results} contains the results with detailed description of how to adopt DevOps. Section \ref{sec:categories_concepts} presents details about categories and related concepts; In section \ref{sec:case_study} is presented a case study that demonstrates the potential application of the results. The paper concludes in section \ref{sec:conclusion}.
