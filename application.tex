\section{Case Study}

Our proposed model was applied to guide the DevOps adoption in the Brazilian Federal Court of Accounts (hereafter TCU), one important government institution. The TCU has the evolution of DevOps as one of the strategic objectives of its area of information technology.

Before the application of the model, TCU was produced some result in deployment automation and the focus was being directed to the tooling issue. Although of some advance in the relation between ops and dev teams, this was not placed as the most important point of DevOps adoption.

Acoes desenvolvidas para incrementar a cultura de colaboracao:
\begin{itemize}
\item BBT sobre cultura de colaboracao DevOps para disseminar essa informacao;
\item Reducao da burocracia na comunicacao entre os times: nada de ticket no servicedesk, canal no slack e aproximacao fisica dos times.
\item Treinamentos tipicos de ops extendidos para o time de dev;
\item Participacao de ambos os times em eventos de DevOps (devops days, por exemplo);
\item Inclusao de DevOps como tema fixo para BBTs;
\item Criacao de comite de arquitetura com representacao de todos os times para promover transparencia e padronizacao na execucao de procedimentos
\end{itemize}
