\section{Hypotheses and Application}

The results of a grounded theory study, as the name of the method itself
suggests, are grounded on the collected data, só the hypotheses emerged from
data. A grounded theory should describe the key relationships between those
categories, i.e. a set of inter-related hypotheses \cite{hoda2017becoming}.
We have presented the main categories of the grounded theory of DevOps adoption
as a network of four categories of enablers (automation, continuous measurement,
sharing and transparency) that are commonly used to develop the collaboration
culture. And this adoption process typically produces three categories of
outcomes (agility, quality assurance and resilience). In this section we
describe the relationships between those categories, i.e. the hypotheses.

\subsection{Relationships Between Categories}

\textbf{H1:} \textit{Existe um conjunto de conceitos que sao tipicamente
utilizados para propiciar o amadurecimento de uma cultura de colaboracao que,
por sua vez, constitui o nucleo da adocao de DevOps}

\textbf{H2:} \textit{Nao ha hierarquia entre os enablers}

\textbf{H3:} \textit{Os outcomes podem ou nao ser explorados}

\subsection{Applying the Theory to direct DevOps adoption on TCU}
A teoria foi aplicada para direcionar a adocao de DevOps no TCU.
