\section{Related Work} \label{sec:related_work}

DevOps is a term that was born in industry. Although academic studies seek
reflect the industry practice, there is a lack of studies investigating it
directly. In this section we describe the studies overlapping with the scope of our work.

A literature review from Erich and colleagues~\cite{cooperation_dev_ops_esem_2014} proposed eight
main concepts related to DevOps: culture, automation, measurement, sharing,
services, quality assurance, structures and standards. The research points out
that four of this concepts (culture, automation, measurement and sharing) are
related to CAMS framework, proposed John Willis~\cite{what_devops_means_2010}, which
became an influential work. The paper concludes that there is a great
opportunity for empirical researchers to study organizations experimenting with
DevOps.

Since then, some studies (e.g.,~\cite{devops_a_definition_xp_15,dimensions_of_devops_xp_15,extending_dimensions_icsea_16},
\cite{characterizing_devops_sbes_2016,qualitative_devops_journalsw_17}) have mixed new literature reviews
with empirical investigations in industry to investigate DevOps.

Although our research and recent literate are interested in understanding DevOps,
there are subtle differences in both (1) the methodological aspects and (2) the focus
of each work. Table~\ref{related_work_table} compares the most relevant literature to
out work. Since the work of Lwakatare et al.~\cite{extending_dimensions_icsea_16} is an
extension of a previous contribution of the authors and her team \cite{dimensions_of_devops_xp_15},
we only compared the most recent study at Table~\ref{related_work_table}.

\begin{table*}[t]
\centering
\caption{Comparing Methodology and Focus of Related Work}
\label{related_work_table}
\begin{tabular}{|p{1.7cm}|p{2.72cm}|p{2.72cm}|p{2.72cm}|p{2.72cm}|p{2.72cm}|}
\toprule
% Cabeçalho
\textbf{Topic} &
\textbf{Our study} &
\textbf{J. Smeds et al.~\cite{devops_a_definition_xp_15}} &
\textbf{Lwakatare et al.~\cite{extending_dimensions_icsea_16}} &
\textbf{Fran\c{c}a et al.~\cite{characterizing_devops_sbes_2016}} &
\textbf{Erich et al.~\cite{qualitative_devops_journalsw_17}} \\ \midrule
% Fim do cabeçalho

Methodology

&
Grounded Theory study interviewing practitioners which contributed to DevOps
adoption in 15 companies across 5 countries.

&
Systematic literature review;
\newline \newline
Semi structured interviews with 13 employees of a single company whose DevOps
process was at an initial stage.

&
Multivocal literature review using data from gray literature;
\newline \newline
Three interviews with software practitioners in one company that was applying
DevOps practices in one project.

&
Multivocal literature review with qualitative analysis procedures from
Grounded Theory.

&
Systematic literature review;
\newline \newline
Interviews with practitioners from 6 companies across 3 countries.


% Linha 2: Focus
\\ \midrule

Focus
&
Explain how DevOps has been successfully adopted in practice;
\newline \newline
Provide guidelines to be used in new DevOps adoptions.

&
Identify in literature the main defining characteristics of DevOps;
\newline \newline
Build a list of possible impediments to DevOps adoption through an empirical
study.

&
Identify how do practitioners describe DevOps as a phenomenon;
\newline \newline
Identify the DevOps practices according to software practitioners.

&
Provide a DevOps definition;
\newline \newline
Identify DevOps practices, required skills, characteristics, benefits and
issues motivating its adoption.

&
Identify how literature defines DevOps;
\newline \newline
Investigate how DevOps is being implemented in practice.\\ \bottomrule

\end{tabular}
\end{table*}

Some of the cited papers claimed for new empirical research focused on
investigating DevOps adoption in companies that have successfully adopt it.
According to J. Smeds et al.~\cite{devops_a_definition_xp_15}, to fully understand and be able to make
generalizable conclusions, further research needs to include several
organizations and to be done on the later stages of the adoption process.
Likewise, Lwakatare et al.~\cite{extending_dimensions_icsea_16} ask for new researches
focusing on further empirical evidence of DevOps practices and patterns in
companies that claim to have implemented it.

Part of the work of Erich and colleagues~\cite{qualitative_devops_journalsw_17}
focused on investigating the
ways in which organizations implement DevOps. However, this contribution was
used to validate data collected using literature review and not to formulate
new hypothesis about DevOps adoption, for instance, using interviews data.

However, none of the aforementioned works focused on explaining the process of adoption DevOps,
in particular, using data collected in industry practice. This is unfortunate since the
practitioners' perception presents an unique point of view that researchers
alone could hardly grasp. Moreover, although the literature have a number of
useful elements, there is a need to complement such elements with a view of how DevOps has
been adopted, containing guidance about how to connect all these isolated parts
and then enabling new candidates to adopt DevOps in a more consistent way.

In terms of results, our main distinct contribution is to improve the guidance
to new practitioners in DevOps adoption. Next, we present the overlaps of our
results with related work (main similarities) and also the main differences.

Comparing with the work of J. Smeds et al.~\cite{devops_a_definition_xp_15}:
\begin{itemize}
\item \textbf{Main Similarities:}
  (1) Grouping elements as DevOps enablers; and
  (2) Similar concepts:\newline
    - Test, deployment, monitoring, recovery and infrastructure automation;\newline
    - Continuous integration, testing and deployment;\newline
    - Service failure recovery without delay;\newline
    - Constant, effortless communication.\newline
\item \textbf{Main Differences:}
(1) Not grouping of concepts, for example: most of their technological enablers
were grouped together in our study in an “automation” category; and (2) Presents
cultural enablers as contributors to DevOps capabilities, not as the most
important concern.
\end{itemize}

Comparing with the work of Lwakatare et al.~\cite{extending_dimensions_icsea_16}
\begin{itemize}
\item \textbf{Main Similarity:}
  All dimensions (collaboration, automation, culture, monitoring and measurement)
  are present in our study.

\item \textbf{Main Differences:}
  (1) Collaboration and culture are presented by us as a single abstraction;
  (2)Concepts related to monitoring and measurement are grouped by us in a single
  category: continuous measurement; and (3) Does not indicate a major dimension.
\end{itemize}

Comparing with the work of Fran\c{c}a et al.~\cite{characterizing_devops_sbes_2016}
\begin{itemize}
\item \textbf{Main Similarities:}
  (1) Automation, sharing, measurement and quality assurance are presented as DevOps
  categories;
  (2) Social Aspects category is similar to our Collaboration Category.

\item \textbf{Main Differences:}
  (1) Presents DevOps as a set of principles, different of enablers and outcomes of
  our study;
  (2) Leanness category is not present in our study and Resilience category is not
  present in theirs; and (3) Does not indicate a core category.

\end{itemize}

Comparing with the work of Erich et al.~\cite{qualitative_devops_journalsw_17}
\begin{itemize}
\item \textbf{Main Similarity:} 5 of their 7 groups are present in our study
(culture of collaboration, automation, measurement, sharing and quality assurance).
\item \textbf{Main Differences:} (1) Does not consolidate practitioners
perspective, only put the it in comparison with literature review results; and
(3) Does not indicate a major group.
\end{itemize}

%\begin{table}[H]
%\centering
%\caption{Comparing Results of Related Work}
%\label{related_work_results_table}
%\begin{tabular}{|p{1.5cm}|p{3cm}|p{3cm}|}
%\toprule
%& \textbf{Main Similarities} & \textbf{Main Differences} \\ \midrule
%
%J. Smeds et al.
%&
%Grouping elements as DevOps enablers; \newline
%
%Similar concepts:\newline
%- Test, deployment, monitoring, recovery and infrastructure automation;\newline
%- Continuous integration, testing and deployment;\newline
%- Service failure recovery without delay;\newline
%- Constant, effortless communication.
%
%&
%Not grouping of concepts, for example: most of their technological enablers
%were grouped together in our study in an “automation” category;\newline
%
%Presents cultural enablers as contributors to DevOps capabilities, not as the
%most important concern.
%
%\\ \midrule
%
%Lwakatare et al.
%&
%All dimensions (collaboration, automation, culture, monitoring and measurement)
%are present in our study.
%
%&
%Collaboration and culture are presented by us as a single abstraction;\newline
%
%Concepts related to monitoring and measurement are grouped by us in a single
%category: continuous measurement;\newline
%
%Does not indicate a major dimension.
%
%\\ \midrule
%
%Fran\c{c}a et al.
%&
%Automation, sharing, measurement and quality assurance are presented as DevOps
%categories;\newline
%
%Social Aspects category is similar to our Collaboration Category.
%
%&
%Presents DevOps as a set of principles, different of enablers and outcomes of
%our study;\newline
%
%Leanness category is not present in our study and Resilience category is not
%present in theirs;\newline
%
%Does not indicate a core category.
%
%\\ \midrule
%
%Eric et al.
%&
%5 of their 7 groups are present in our study (culture of collaboration,
%automation, measurement, sharing and quality assurance).
%
%&
%Does not consolidate practitioners perspective, only put the it in comparison
%with literature review results;
%
%Does not indicate a major group.
%
%\\
%
%\bottomrule
%\end{tabular}
%\end{table}
