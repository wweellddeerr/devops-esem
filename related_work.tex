\section{Related Work}

A literature review of 2014 \cite{cooperation_dev_ops_esem_2014} proposed eight
main concepts related to DevOps: culture, automation, measurement, sharing,
services, quality assurance, structures and standards.
The research points out that four of this concepts (culture, automation,
measurement and sharing) are related to CAMS framework, proposed in
\cite{what_devops_means_2010}, which multiple articles referred to.
The paper concludes that there is a great opportunity for empirical researchers
to study organizations experimenting with DevOps.

Since then, some studies have mixed literature reviews with empirical
investigations in the industry \cite{devops_a_definition_xp_15},
\cite{a_qualitative_study_journal_sw_17} and \cite{dimensions_of_devops_xp_15}.

In \cite{dimensions_of_devops_xp_15}, is presented a framework with four main
dimensions that characterize DevOps: collaboration, automation, measurement and
monitoring. The study advocates to confirm three elements of DevOps presented
in previous research and to add a new element (monitoring). The paper doesnt
indicates the role of each dimension in a DevOps adoption, causing some gaps:
DevOps only make sense with all dimensions being improved at the same time?
Does evolving into any single dimension provide proper adoption way of DevOps?
If automation, by example, is fully attended, is the DevOps adoption in 25\% of
progress? Depite of theses gaps, the paper offers an abrangent view of DevOps
and the results combine literature review with a short direct investigation of
industry through interviews with three practitioners. Except monitoring that
was considered part of continuous measurement, the another dimensions are
present too in our results, which validates them. The paper also concludes
that still a need for empirical research that investigates DevOps adoption in
order to validate and enhance the presented conceptual framework.

One extension of \cite{dimensions_of_devops_xp_15} was proposed in
\cite{extending_dimensions_icsea_16}. This extension added culture as a new
dimension of DevOps characterization and presented practices related to each
dimension. The data was collected using multivocal literature review combined
with three semi-structured interviews. The data analysis was performed using
coding techniques similar to those used in Grounded Theory studies. In our study
culture is grouped with collaboration in one category named \"collaboration
culture\", so this five dimensions are presented in our results as only three,
but with similar descriptions. The paper present as future work the need to
obtain empirical evidence of DevOps adoption in companies that claim to have
implemented it, exactly as we are proposing here.

Another source of useful elements that can help in a DevOps adoption is
presented in \cite{characterizing_devops_sbes_2016}. Applying some Grounded
Theory data analysis techniques in data collected from gray literature, some
results are presented like: .

Parágrafo sobre \cite{a_qualitative_study_journal_sw_17}

DevOps is a term that was born in industry. Although the academic studies seek
reflect the industry practice, there is a lack of studies investigating it
directly. None of the papers cited above focused on directly investigate
industry practice and most of them claimed for new research of this type.
