\section{Related Work}

A literature review of 2014 \cite{cooperation_dev_ops_esem_2014} proposed eight
main concepts related to DevOps: culture, automation, measurement, sharing,
services, quality assurance, structures and standards.
The research points out that four of this concepts (culture, automation,
measurement and sharing) are related to CAMS framework, proposed in
\cite{what_devops_means_2010}, which multiple articles referred to.
The paper concludes that there is a great opportunity for empirical researchers
to study organizations experimenting with DevOps.

Since then, some studies have mixed literature reviews with empirical
investigations in the industry \cite{devops_a_definition_xp_15},
\cite{a_qualitative_study_journal_sw_17} and \cite{dimensions_of_devops_xp_15}.

In \cite{dimensions_of_devops_xp_15}, is presented a framework with four main
dimensions that characterize DevOps: collaboration, automation, measurement and
monitoring. The study advocates to confirm three elements of DevOps presented
in previous research and to add a new element (monitoring). The paper doesnt
indicates the role of each dimension in a DevOps adoption, causing some gaps:
DevOps only make sense with all dimensions being improved at the same time?
Does evolving into any single dimension provide proper adoption way of DevOps?
If automation, by example, is fully attended, is the DevOps adoption in 25% of
progress? Depite of theses gaps, the paper offers an abrangent view of DevOps
and the results combine literature review with interviews with three
practitioners. Except monitoring that was considered part of continuous
measurement, the another dimensions are present too in our results, which
validates them.


DevOps is a term that was born in industry. Although the academic studies seek
reflect the industry practice, there is a lack of studies investigating it
directly. None of the papers cited above focused on directly investigate
industry practice and most of them claimed for new research of this type.
