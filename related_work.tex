\section{Related Work} \label{sec:related_work}

A literature review by Erich et al.~\cite{cooperation_dev_ops_esem_2014} presents 8
main concepts related to DevOps: culture, automation, measurement, sharing,
services, quality assurance, structures and standards. The authors pointed out
that the first four concepts are
related to the CAMS framework, proposed by Willis~\cite{what_devops_means_2010}.
The paper concludes that there is a great opportunity for empirical researchers
to study organizations experimenting with DevOps.
Other studies (e.g.,~\cite{devops_a_definition_xp_15,dimensions_of_devops_xp_15,extending_dimensions_icsea_16,characterizing_devops_sbes_2016,qualitative_devops_journalsw_17})
mixed literature reviews with empirical data to investigate DevOps.
Although our research and recent literate are interested in understanding DevOps,
there are subtle differences in both (1) the methodological aspects and (2) the focus
of each work.

%% Table~\ref{related_work_table} compares the most relevant literature to
%% out work. Since the work of Lwakatare et al.~\cite{extending_dimensions_icsea_16} is an
%% extension of a previous contribution of the authors and her team \cite{dimensions_of_devops_xp_15},
%% we only compared the most recent study at Table~\ref{related_work_table}.

%% \begin{table*}[t]
%% \centering
%% \caption{Comparing Methodology and Focus of Related Work}
%% \label{related_work_table}
%% \begin{tabular}{|p{1.7cm}|p{2.72cm}|p{2.72cm}|p{2.72cm}|p{2.72cm}|p{2.72cm}|}
%% \toprule
%% % Cabeçalho
%% \textbf{Topic} &
%% \textbf{Our study} &
%% \textbf{J. Smeds et al.~\cite{devops_a_definition_xp_15}} &
%% \textbf{Lwakatare et al.~\cite{extending_dimensions_icsea_16}} &
%% \textbf{Fran\c{c}a et al.~\cite{characterizing_devops_sbes_2016}} &
%% \textbf{Erich et al.~\cite{qualitative_devops_journalsw_17}} \\ \midrule
%% % Fim do cabeçalho

%% Methodology

%% &
%% Grounded Theory study interviewing practitioners which contributed to DevOps
%% adoption in 15 companies across 5 countries.

%% &
%% Systematic literature review;
%% \newline \newline
%% Semi structured interviews with 13 employees of a single company whose DevOps
%% adoption process was at an initial stage.

%% &
%% Multivocal literature review using data from gray literature;
%% \newline \newline
%% Three interviews with software practitioners in one company that was applying
%% DevOps practices in one project.

%% &
%% Multivocal literature review with qualitative analysis procedures from
%% Grounded Theory.

%% &
%% Systematic literature review;
%% \newline \newline
%% Interviews with practitioners from 6 companies across 3 countries.


%% % Linha 2: Focus
%% \\ \midrule

%% Focus
%% &
%% Explain how DevOps has been successfully adopted in practice;
%% \newline \newline
%% Provide guidelines to be used in new DevOps adoptions.

%% &
%% Identify in literature the main defining characteristics of DevOps;
%% \newline \newline
%% Build a list of possible impediments to DevOps adoption through an empirical
%% study.

%% &
%% Identify how do practitioners describe DevOps as a phenomenon;
%% \newline \newline
%% Identify the DevOps practices according to software practitioners.

%% &
%% Provide a DevOps definition;
%% \newline \newline
%% Identify DevOps practices, required skills, characteristics, benefits and
%% issues motivating its adoption.

%% &
%% Identify how literature defines DevOps;
%% \newline \newline
%% Investigate how DevOps is being implemented in practice.\\ \bottomrule

%% \end{tabular}
%% \end{table*}

%% Some of the cited papers claims for new empirical research focused on
%% investigating DevOps adoption in companies that have successfully adopt it.
%% According to J. Smeds et al.~\cite{devops_a_definition_xp_15}, to fully understand and be able to make
%% generalizable conclusions, further research needs to include several
%% organizations and to be done on the later stages of the adoption process.
%% Likewise, Lwakatare et al.~\cite{extending_dimensions_icsea_16} ask for new researches
%% focusing on further empirical evidence of DevOps practices and patterns in
%% companies that claim to have implemented it.

First of all, none of the aforementioned works focused on explaining the process of adoption DevOps,
in particular, using data collected in industry. This is unfortunate since the
practitioners' perception present an unique point of view that researchers
alone could hardly grasp. Moreover, although the literature have a number of
useful elements, there is a need to complement such elements with a perspective on how DevOps has
been adopted, containing guidance about how to connect all these isolated parts
and then enabling new candidates to adopt DevOps in a more consistent way.
For instance, the work of Erich et al.~\cite{qualitative_devops_journalsw_17}
focus on investigating the ways in which organizations implement DevOps.
However, this work relies only in literature review and does not formulate
new hypothesis about DevOps adoption. Second,
in terms of results, our main distinct contribution is to improve the guidance
to new practitioners in DevOps adoption.
Next, we present the overlappings of our
results with the existing literature, presenting also the main differences that
make the contributions of our work more clear.

The work of J. Smeds et al.~\cite{devops_a_definition_xp_15} uses a literature
review to produce one explanation about DevOps through a set of technological and
cultural enablers. Additionally, their results
present a set of impediments of DevOps adoption based on an interview with 13
subjects of a same company, and whose DevOps adoption process was at
an initial stage. The main similarities with our study are: (1) grouping
elements as DevOps enablers; and (2) the presence of several similar concepts:
(a) test, deployment, monitoring, recovering and infrastructure automation;
(b) continuous integration, testing and deployment; (c) service failure recovery
without delay; and (d) constant, effortless communication. While the main
differences are: (1) their work does not group concepts into categories,
for example: most of their technological enablers were grouped together in our
study within the \cat{automation} category; (2) presents cultural enablers as
common contributor to DevOps, not as the most important concern; and (3) the empirical
part of the study focus on building a list of possible impediments to DevOps
adoption, not on provide guidance to new adopters.

Lwakatare et al.~\cite{extending_dimensions_icsea_16} proposed a conceptual
framework to explain ``DevOps as a phenomenon". The framework is organized around
five dimensions (collaboration, automation, culture, monitoring and measurement)
and these dimensions are presented with related practices. The main similarity
with our study is that all dimensions are also presented here. While the
main differences are: (1) collaboration and culture are presented by us
as a single abstraction; (2) Concepts related to monitoring and measurement are
grouped by us in a single category: \cat{continuous measurement}; and (3) Does
not indicate a major dimension (aka, the core category).

Fran\c{c}a et al.~\cite{characterizing_devops_sbes_2016} present a DevOps
explanation produced using a multivocal literature review. The data was collected
from multiple sources, including gray literature, and analyzed using procedures
from GT. The results contain a set of DevOps principles, where
there is most of the overlapping with our study. Beyond these principles, the paper
presents a definition to DevOps, issues motivating its adoption, required skills,
potential benefits and challenges of adopting DevOps. The main similarities
are: (1) Automation, sharing, measurement and quality assurance are presented as
DevOps categories; and (2) Their social aspects category is similar to our
\cc category. While the main differences are: (1) Presents DevOps as a
set of principles, different of enablers and outcomes our study; and (2) Leanness
category is not present in our study and \cat{resilience} category is not present
in theirs; and (3) Does not indicate a core category.

The study conducted by Erich et al.~\cite{qualitative_devops_journalsw_17},
similarly to the others cited above, combined literature review with some
interviews with practitioners. In the literature review part, the papers were
labeled and the similar labels are grouped. The 7 top labels are then presented
as elements of DevOps usage in literature: culture of collaboration, automation,
measurement, sharing, services, quality assurance and governance. After literature
review, six interviews were conducted in order to obtain evidence of DevOps
adoption in practice. The interviews were analyzed individually and a comparison
between them was made focusing on problems that organizations try to solve by
implementing DevOps, problems encountered when implementing DevOps and practices
that are considered part of DevOps. The main similarity with our study
is that 5 of their 7 groups are also present in our study (culture of collaboration,
automation, measurement, sharing and quality assurance). While the main
differences are: (1) Does not consolidate practitioners perspective, only put it
in comparison with literature review results; and (3) Does not indicate a major group.
