\section{Related Work} \label{sec:related_work}

A literature review of 2014 \cite{cooperation_dev_ops_esem_2014} proposed eight
main concepts related to DevOps: culture, automation, measurement, sharing,
services, quality assurance, structures and standards. The research points out
that four of this concepts (culture, automation, measurement and sharing) are
related to CAMS framework, proposed in \cite{what_devops_means_2010}, which
multiple articles referred to. The paper concludes that there is a great
opportunity for empirical researchers to study organizations experimenting with
DevOps.

Since then, some studies \cite{devops_a_definition_xp_15},
\cite{dimensions_of_devops_xp_15}, \cite{extending_dimensions_icsea_16},
\cite{characterizing_devops_sbes_2016} and
\cite{qualitative_devops_journalsw_17} have mixed new literature reviews
with empirical investigations in industry to investigate DevOps.

Although our research and related work have investigated the same area, which
is DevOps, there are subtle differences in methodological aspects and in focus
of each work. Table~\ref{related_work_table} summarizes methodology and focus
of each work, including our. \cite{extending_dimensions_icsea_16} is one
extension of \cite{dimensions_of_devops_xp_15}, so, we only put the new one
on the table.

\begin{table*}[t]
\centering
\caption{Comparing Methodology and Focus of Related Work}
\label{related_work_table}
\begin{tabular}{|p{1.4cm}|p{2.72cm}|p{2.72cm}|p{2.72cm}|p{2.72cm}|p{2.72cm}|}
\toprule
% Cabeçalho
\textbf{Topic} &
\textbf{Our study} &
\textbf{Jens et al.~\cite{devops_a_definition_xp_15}} &
\textbf{Lwakatare et al.~\cite{extending_dimensions_icsea_16}} &
\textbf{Fran\c{c}a et al.~\cite{characterizing_devops_sbes_2016}} &
\textbf{Erich et al.~\cite{qualitative_devops_journalsw_17}} \\ \midrule
% Fim do cabeçalho

Methodology

&
Grounded Theory study interviewing practitioners which contributed to DevOps
adoption in 15 companies across 5 countries.

&
Systematic literature review;
\newline \newline
Semi structured interviews with 13 employees of a single company whose DevOps
process was at an initial stage.

&
Multivocal literature review using data from gray literature;
\newline \newline
Three interviews with software practitioners in one company that was applying
DevOps practices in one project.

&
Multivocal literature review with qualitative analysis procedures from
Grounded Theory.

&
Systematic literature review;
\newline \newline
Interviews with practitioners from 6 companies across 3 countries.


% Linha 2: Focus
\\ \midrule

Focus
&
Explain how DevOps has been successfully adopted in practice;
\newline \newline
Provide guidelines to be used in new DevOps adoptions.

&
Identify in literature the main defining characteristics of DevOps;
\newline \newline
Build a list of possible impediments to DevOps adoption through an empirical
study.

&
Identify how do practitioners describe DevOps as a phenomenon;
\newline \newline
Identify the DevOps practices according to software practitioners.

&
Provide a DevOps definition;
\newline \newline
Identify DevOps practices, required skills, characteristics, benefits and
issues motivating its adoption.

&
Identify how literature defines DevOps;
\newline \newline
Investigate how DevOps is being implemented in practice.\\ \bottomrule

\end{tabular}
\end{table*}

Some of the cited papers claimed for new empirical research focused on directly
investigate DevOps adoption in companies that advocates adopt it. To Jens et al.
~\cite{devops_a_definition_xp_15}, to fully understand and be able to make
generalizable conclusions, further research needs to include several
organizations and to be done on the later stages of the adoption process.
Lwakatare et al.~\cite{extending_dimensions_icsea_16} ask for new researches
focusing on further empirical evidence of DevOps practices and patterns in
companies that claim to have implemented it.

One part of \cite{qualitative_devops_journalsw_17} focused on investigating the
ways in which organizations implement DevOps. But, that part of the paper was
used to validate data collected using literature review and not to formulate
a new explanation about DevOps adoption using interviews data.

DevOps is a term that was born in industry. Although the academic studies seek
reflect the industry practice, there is a lack of studies investigating it
directly. None of the papers cited above focused on explain DevOps adoption
using data collected in industry practice. The presented work has a number of
useful elements, but that need to be complemented with a view of how DevOps has
been adopted, containing guidance about how to connect all these isolated parts
and then enabling new candidates to adopt DevOps in a more consistent way.
